\documentclass[11pt]{article}
%\renewcommand\refname{ }

\usepackage{fullpage}
\usepackage{epsfig}
\usepackage{graphicx}
\usepackage{listings,color}
\usepackage{mdframed}
\usepackage{float}
\parskip=10pt
\setlength\parindent{0pt}


\input macros.tex

\definecolor{quotegray}{gray}{0.9}
% \lstnewenvironment{referee}{%
%   \lstset{backgroundcolor=\color{quotegray},
%   frame=single,
%   framerule=0pt,
%   basicstyle=\ttfamily,
%   columns=fullflexible}}{}

% https://tex.stackexchange.com/questions/111065/quoting-styles-technical-an-appreciation-questions
\mdfdefinestyle{myquotestyle}{
  leftmargin=15pt,
  rightmargin=15pt,
  backgroundcolor=quotegray,
  linewidth=0pt,
  skipbelow=\topskip,
  skipabove=\topskip
}
\newenvironment{referee}[1][]{%
    \ignorespaces%
    \begin{mdframed}[style=myquotestyle,#1]%
}{%
    \end{mdframed}%
    \ignorespacesafterend%
}%

% \newenvironment{referee}[1][]{
%   \ignorespaces%
%   \begin{mdframed}[style=myquotestyle,#1]%
% }{%
%   \end{mdframed}%
%   \ignorespaceafterend%
% }%



\begin{document}

\begin{center} 
\bfseries{
\begin{large}
  Response to referee report for manuscript ref. MN-19-2054-MJ.R1
\end{large}
}
\end{center}

We thank the referee for their response to our recent revision. Their review is directly quoted below in the gray boxes with our responses below.  Any changes to the manuscript, have been \textbf{bold-faced}.

\begin{referee}
    The authors have clarified most of the open questions in the revised manuscript. A few points need further discussion, they are listed below: 
\end{referee}

\begin{referee}
    In this paper, some very long paragraphs can be found, especially in Sections 3 and 4. Please break these long paragraphs down in smaller ones to increase the read flow. 
\end{referee}
    These paragraphs have been broken down. 

\begin{referee}
    To Major comment number 1) 

    The convergence study is a great addition to the paper. However, the lowest limit explored here (0.625pc) is a factor of roughly 100 larger than the scale on which fragmentation can be seen (compare eg the 1000AU regions, in which several Pop III stars form in work by Stacy+16). I recommend to explicitly state down to which resolution the convergence was tested in Section 2.2.1 and adding these limitations to the "Caveats" section. Instead of single Pop III stars, this paper is a study of the multiplicity of Pop III star formation regions. Please make this clear in the paper. 
\end{referee}
    In Section 2.2, we have expanded our description of the convergence study to include the smallest resolution and pointed out that the overdensity formation criterion is similar to a molecular core.  We explicitly state that our results are lower limits on the number of Pop III stars, massive and low-mass, in this section.  Furthermore, we remind the reader about this lower limit in Section 3.3, where we present our results on the stellar multiplicity.  Lastly, we have added a paragraph in the Caveats section (\S4.3) about our simulation not resolving fragmentation below 0.625 pc and conclude that our work strengthens the case for multiple Pop III stars in each halo, especially considering this limitation.
    
    That being said, we decided to retain the naming of these star forming regions as ``stars'' throughout the paper because the phrase ``star forming regions'' or something similar would have been too wordy as it is mentioned many times throughout the manuscript and in the figures.  We feel that emphasizing that our results are lower limits on the number of massive Pop III stars throughout the paper is a good compromise.

\begin{referee}
    Minor points: 

    - Is the SFRD really in physical (and not comoving) units (Figure 4 / Section 3)? Please provide values for the comparison papers (Xu+13, Magg+16) and include a comparison to Jaacks+19. 
\end{referee}
    Upon review of these calculations, you are correct in questioning the units of the SFRD. The SFRD is in comoving units. This correction has been added to the caption of the SFRD and to the description above the plot. 

    These comparison values have been added to the beginning of the results section.

\begin{referee}
    - The initialization at z=130 should be mentioned as a caveat. 
\end{referee}
	We have added the following text to Section 2.1 after stating the initialization redshift, ``We note that this initialization redshift is too low, even with 2nd order Lagrangian perturbation theory, for a small cosmological volume and results in delayed structure formation at very high redshift $z > 20$.''    


\begin{referee}
    - The analytic calculation of the self-shielding of the halo now takes into account a realistic halo profile. However, there are three issues: 
    -- In the first paragraph of Section 2.4, it still reads "with a constant H2 fraction". Please correct this. 
\end{referee}
    This has been corrected.

\begin{referee}
    -- For completeness, please provide the readers with the H2 radial fit to the O'Shea \& Normal 07 paper that was performed. 
\end{referee}
    The citation previously written was referencing their first paper in their series of Pop III star formation. It should have referenced their second paper in the series. This has been corrected.
    
    Our fitting formula has been added to the text where their H2 profiles are discussed.


\begin{referee}
    -- Compared to the previous version with a constant profile, the shielding factor increases for higher halo masses (in the previous version of the paper, there was a slight decrease): leading to less shielding. I do not understand where this is coming from, a more massive halo has a larger overall gas mass, and therefore more H2 and more shielding. Maybe it has to do with the scaling of the H2 fraction?  It is important to know how the authors scale the H2 fraction as a function of radius for different halo masses (see point above). If this is not the reason, please explain what else is causing more massive haloes to be worse at self-shielding.  
\end{referee}
    We thank the reader for asking these questions. They have caused us to go back to our calculations to try to find the source of the increase in the shielding factor for larger halo masses. It turned out that there was an error in our calculations stemming from when we assumed a constant H2 fraction. This has been fixed, as well as an incorrect decimal place in our core radius calculations, and we now see decreased shielding factors (more effective shielding) for increased halo masses, as expected. We would like to remind the referee that the dashed lines indicate a smaller core radius, and thus, the shielding factor is even more effective for this core radius since more H2 is being added to the column density. 

\begin{referee}
    - Thank you for adding the global LWBG as a dashed line in Figure 3. Please also mention at some point in the text or add a legend that this is again given in units of J21. 
\end{referee}
    The solid red line in the legend refers to both the solid and the dashed red lines. This detail has been added to the caption and to the text. 

\begin{referee}
    - I see that retrospectively calculating the time-averaged LWBG is not possible without repeating the study. However, please mention the possibility for a correlation between the time-integrated LWBG and the host halo mass in Section 3.2 and summary and be more explicit that this is the LWBG at star formation (eg in the introduction and in Section 3.2 and in the second bullet point of the Conclusions).  
\end{referee}
    The possible correlation between the LW background and the host halo mass has been added to the requested sections.

    We have also added the detail about how we are investigating the LWBG at the instance of star formation to the requested sections.

\begin{referee}
    - The authors put their work in comparison to Schauer+19, ApJL 877, 5 who match the star formation efficiency in minihaloes to the observed 21cm signal. They should rather compare to Schauer+19 MNRAS, 484, 3510 who performed cosmological hydrodynamical simulations for the minimum and average halo mass in different streaming velocity regions. Please include their limits in your Figure 10, as done with the Naoz+13 results, and a short comparison to Hirano+18 (their Science paper). Please include these references also in the introduction. Also the sentence "Often in the literature, streaming velocities are only studied for a few haloes" seems outdated.
\end{referee}
    We have added a panel to the left of Figure 10 to de-clutter the main figure and to separate the halo mass ranges and minima related to streaming velocities. 
    
    The limits from Schauer+19 have been added to the left panel of Figure 10, and the previous description regarding the 21cm signal has been removed. Their citation has been added to the list of citations investigating streaming velocities. 

    We have added a short comparison to Hirano+17 (Science) and Schauer+19 have also added their citation to the list of citations mentioned previously. 
    
    These citations have also been added to the introduction.

	The sentence "Often in the literature, streaming velocities are only studied for a few haloes" has been removed.
    
\end{document}
