\documentclass[11pt]{article}
%\renewcommand\refname{ }

\usepackage{fullpage}
\usepackage{epsfig}
\usepackage{graphicx}
\usepackage{listings,color}

\input macros.tex

\definecolor{verbgray}{gray}{0.9}
\lstnewenvironment{referee}{%
  \lstset{backgroundcolor=\color{verbgray},
  frame=single,
  framerule=0pt,
  basicstyle=\ttfamily,
  columns=fullflexible}}{}

\begin{document}

\begin{center} 
\bfseries{
\begin{large}
  Response to referee report for manuscript ref. MN-19-2054-MJ
\end{large}
}
\end{center}

\begin{referee}
The paper presented here is very well written and offers important new insights. It studies the dependence of the halo mass for first star formation on a Lyman-Werner background in a large simulation, and goes beyond current literature work by the inclusion of self-shielding and studying the multiplicity of Pop III stars. I generally recommend if for publication after some concerns are addressed.
\end{referee}


\begin{referee}
Major comments:
1) One of the main points of the paper is studying the multiplicity of Pop III stars in a halo. This multiplicity might be due to the fact that Pop III stars have collapsed to a BH after their lifetime and a second Pop III star could form in the same halo. This is very plausible. If several Pop III stars are found forming in the same halo at the same time, however, the employed star particle algorithm has a hard-coded length scale of 1pc in which only one star particle can be formed. I am wondering if this is an effect that is resolution dependent - do the same number of stars form if the resolution is increased and the maximum distance is decreased? Which is the motivation for the length scale of 1pc?
\end{referee}

-Resolution testing



\begin{referee}
2) The authors select their haloes by the halo finding algorithm Rockstar. They should add a bit of information on that code (is it fof or subfind-like?). Also, one of their main points is to compare their work to Machacek+01. Machacek+01, however, use a different halo definition (spherical around the halo centre going out to a radius of 200 times the mean density of the Universe). How do these different halo masses compare?
Further, it is not entirely clear why atomic cooling haloes are excluded in this analysis and if they are excluded throughout the whole paper. The exclusion of atomic cooling haloes artificially shifts the average halo mass to lower values.
In addition, the authors should state the fraction of haloes that have Pop III stars forming outside the halo.
The baryon fraction varies with redshift. Please give an estimate by how much this influences the halo masses used here.
\end{referee}

-address concern about Rockstar details, and Machacek halo definition
-exclusion of atomic cooling haloes - we excluded these because of a bug in our code... how to explain this without simply saying that?
-fraction of halos that halo Pop III stars forming outside the haloes: I'm not sure I will be able to tell exactly how many are not in halos. I need to look back at the data to see if I will be able to pick out which halos I artificially drew a halo around. 
-shall I look at the baryon fraction as a function of time to see how much it varies? 

\begin{referee}
Minor comments:
The authors should state at one point in the paper for which properties they use comoving or physical units.
\end{referee}
-We use physical units throughout the whole paper, apart from when we discuss the boxsize. I will make this more clear. 

\begin{referee}
Introduction:
The authors write that Pop III star formation is generally more massive. While a fraction of the community agrees with that, there is manifoldly cited work that finds low-mass Pop III progenitors, and the authors should not let that unmentioned, but also cite eg Stacy+16, Greif+11 and Clark+11.
\end{referee}
-I will look at and add these citations and detail to the introduction.

\begin{referee}
The authors name a few processes for suppressing Pop III star formation. For completeness, they should also mention eg X-ray heating (citing eg Jeon+14). 
\end{referee}
-I will add this detail

\begin{referee}
The authors cite the well-known study from Machacek+01. However, a more recent result would be from the study by Visbal+14, giving a lower mass limit. Since the authors find a lower mass limit than Machacek+01, it would be interesting to see how this compares to the lower mass limit from the more modern paper.
\end{referee}
-I will check out Visbal+ 14 (I think I have looked at this paper, will double check)

\begin{referee}
Methods:
In Wise+12, three different environment regions are picked. Please provide more description on the rare peak simulation here and discuss why this one is appropriate in this context.
\end{referee}
-I will look at that simulation again, if John could provide some insight here that would be nice.

\begin{referee}
The initialization at z=130 is very late. I am aware that it is impossible to repeat this simulation starting at an earlier redshift, but would advise for future work.
\end{referee}
-Personal question: why is this initialization late?

\begin{referee}
Why do the authors pick a non-zero metalicity threshold for Pop III star formation? How many Pop III stars form between Z=0 and Z=5e-6 Z_sol?
\end{referee}
-Simulation question


\begin{referee}
What is the physical motivation for a density threshold of 1.e6/cc? Is there convergence? 
\end{referee}
-Simulation question

\begin{referee}
Similarly, why do the authors chose a H2 fraction of 1.e-3? (Is is a mass fraction or a number fraction?) How many Pop III stars do not form because of this threshold?
\end{referee}
-Simulation question

\begin{referee}
Do the authors set fixed mass limits to the IMF and if yes, which are they? If not, which is the highest / lowest mass they find in their random sampling? And which fraction of the Pop III stars in the simulation end up as type II SNe / BHs / PISN?
\end{referee}
-We do set fixed mass limits, between 1 and 300 Msol. I will add this detail. 
-What is the best way to get this information? Shall I look at all the Pop III stars at the end of the simulation and look at their masses? I should be able to get a fraction of these endings from that, correct?

\begin{referee}
One effect than can weaken the shielding is shifting out of the Lyman-Werner resonance lines (see Hartwig+15). I am wondering how much this would affect the study presented here. 
\end{referee}
-Will have to look at Hartwig+15. 

\begin{referee}
In Figure 1, please mention in the caption or at the top / bottom of the left axis that a shielding factor of 1 corresponds to no shielding and a shielding factor of 0 corresponds to high shielding, as these nomenclatures are not immediately understandable for the reader.
\end{referee}
-This is a good point, that is confusing. I will add this.

\begin{referee}
The analytic calculation of the shielding factor is a nice property of the paper. However, assuming a constant H2 fraction seems not appropriate. Please repeat this exercise for a more realistic halo profile, which could be found in the simulations or in the literature, eg by Machacek+01 themselves.
\end{referee}
-I will check out a more realistic H2 fraction for a halo and try to get a nice plot out of that.

\begin{referee}
Results:
Figure 2 is quite empty, it would be useful to include the halo mass function of the earliest Pop III formation redshift with its analytic prediction. In addition, the authors could provide a line or an arrow to show their halo mass resolution limit.
\end{referee}
-Figure 2 is empty. I will add a vertical line at 10^5 Msol to show our resolution limit. As far as the HMF of the earliest Pop III formation redshift, is this referring to the HMF at the time when Pop III stars first begin to form, so at about z=27, what is the HMF at that time and how does it compare with Sheth-Tormen? I will plot this up.

\begin{referee}
Figure 3 looks very nice. It would be helpful if an additional line that only provides the global LWBG (in units of J21) was included.
\end{referee}
-Woohoo! Figure 3 does look nice! I will add the global LWBG, hopefully it looks alright. I am fairly certain we did this and for some reason it all didn't seem to add up. I will try this again.

\begin{referee}
The authors find no correlation between the host halo mass and the Lyman-Werner background intensity. However, they only look at the LWBG flux at the time of Pop III formation - is there maybe a correlation between the time integrated LW flux?
\end{referee}
-Haven't thought about the time integrated LW flux. Would like some advice on this point. 

\begin{referee}
Figures 7 and 8 show very nice histograms as an additional panel on the right. I suggest to include a histogram on the spread of the creation time in Figure 9 similar to Figures 7 and 8.
\end{referee}
-I will add a histogram on the spread of the creation time in Figure 9.

\begin{referee}
Discussion:
The authors give three reasons for the halo mass spread. Can they estimate which of the three processes (BH formation, dynamical heating and temporal LWBG fluctuations) is responsible for the spread by which fraction?
\end{referee}
-It's not immediately obvious to me how we would be able to estimate this. I would have to look at these processes in more detail to see if there are some mass ranges correlated with them. Any insight from John?

\begin{referee}
One of the most extensive studies on over 1000 minihaloes was performed by Hirano+15, who also include a Lyman-Werner background. Their work should be put in comparison as well.
\end{referee}
-I will check out this paper, and see what can be added.

\begin{referee}
In Figure 10, the result from Yoshida+03 is hard to see, maybe use a different symbol / colour.
\end{referee}
-I will adjust this symbol.

\begin{referee}
Since the authors study star formation in minihaloes, it would be appropriate to also mention Kitayama+04 and Schauer+15 for the Lyman-Werner escape fractions from minihaloes.
\end{referee}
-I will check out these papers and add accordingly.

\begin{referee}
The authors mention a few times that streaming velocities are another pathway of suppressing Pop III formation (also in the introduction). While they cite some older work, they however fail to compare to more recent work eg by Hirano+18 or Schauer+19 (who do perform a systematic study not only based on a few, single haloes).
\end{referee}
-I will check out these papers as well, and see if I can compare. 

\end{document}