\documentclass[11pt]{article}
%\renewcommand\refname{ }

\usepackage{fullpage}
\usepackage{epsfig}
\usepackage{graphicx}
\usepackage{listings,color}
\usepackage{mdframed}
\parskip=10pt

\input macros.tex

\definecolor{quotegray}{gray}{0.9}
% \lstnewenvironment{referee}{%
%   \lstset{backgroundcolor=\color{quotegray},
%   frame=single,
%   framerule=0pt,
%   basicstyle=\ttfamily,
%   columns=fullflexible}}{}

% https://tex.stackexchange.com/questions/111065/quoting-styles-technical-an-appreciation-questions
\mdfdefinestyle{myquotestyle}{
  leftmargin=15pt,
  rightmargin=15pt,
  backgroundcolor=quotegray,
  linewidth=0pt,
  skipbelow=\topskip,
  skipabove=\topskip
}
\newenvironment{referee}[1][]{%
    \ignorespaces%
    \begin{mdframed}[style=myquotestyle,#1]%
}{%
    \end{mdframed}%
    \ignorespacesafterend%
}%

% \newenvironment{referee}[1][]{
%   \ignorespaces%
%   \begin{mdframed}[style=myquotestyle,#1]%
% }{%
%   \end{mdframed}%
%   \ignorespaceafterend%
% }%



\begin{document}

\begin{center} 
\bfseries{
\begin{large}
  Response to referee report for manuscript ref. MN-19-2054-MJ
\end{large}
}
\end{center}

We thank the referee for their insightful review.  Their review is directly quoted below in the gray boxes with our responses below.  Any changes to the manuscript has been \textbf{bold-faced}.

\begin{referee}
The paper presented here is very well written and offers important new insights.  It studies the dependence of the halo mass for first star formation on a Lyman-Werner background in a large simulation, and goes beyond current literature work by the inclusion of self-shielding and studying the multiplicity of Pop III stars. I generally recommend if for publication after some concerns are addressed.
\end{referee}


\begin{referee}
Major comments:

1) One of the main points of the paper is studying the multiplicity of Pop III stars in a halo. This multiplicity might be due to the fact that Pop III stars have collapsed to a BH after their lifetime and a second Pop III star could form in the same halo. This is very plausible. If several Pop III stars are found forming in the same halo at the same time, however, the employed star particle algorithm has a hard-coded length scale of 1pc in which only one star particle can be formed. I am wondering if this is an effect that is resolution dependent - do the same number of stars form if the resolution is increased and the maximum distance is decreased? Which is the motivation for the length scale of 1pc?
\end{referee}

-Resolution testing
-Need to update enzo to John's version with the bug fix. 

-Restart from DD0029 after pulling it back down. "enzo.exe -r output"
-At the last output, check to see how many stars are in the halo. 
-"our results are insensitive to the star combine radius"


\begin{referee}
2) The authors select their haloes by the halo finding algorithm Rockstar. They should add a bit of information on that code (is it fof or subfind-like?). Also, one of their main points is to compare their work to Machacek+01. Machacek+01, however, use a different halo definition (spherical around the halo centre going out to a radius of 200 times the mean density of the Universe). How do these different halo masses compare?

Further, it is not entirely clear why atomic cooling haloes are excluded in this analysis and if they are excluded throughout the whole paper. The exclusion of atomic cooling haloes artificially shifts the average halo mass to lower values.

In addition, the authors should state the fraction of haloes that have Pop III stars forming outside the halo. The baryon fraction varies with redshift. Please give an estimate by how much this influences the halo masses used here.
\end{referee}

-Added a bit about what Rockstar is. 
-Machacek halo definition comparison - I have plotted the rockstar mass versus the calculated mass, looks like a solid group of halos lie in a 1:1 relationship, and others are fairly close, apart from a group where the calculated mass is much smaller than the rockstar mass. Need to determine why this is.

-Exclusion of atomic cooling haloes - we excluded these because of a bug in our code... how to explain this without simply saying that?

-Fraction of halos that halo Pop III stars forming outside the halos: Turns out I was not able to get this data as easily as I thought. I am going to quickly run through my script and find the P3 stars in each dataset which are not in halos.

-I am waiting to get this data!

\begin{referee}
Minor comments:

The authors should state at one point in the paper for which properties they use comoving or physical units.
\end{referee}
-Clarified this at the end of the second paragraph in the Simulation Setup section.

\begin{referee}
Introduction:

The authors write that Pop III star formation is generally more massive. While a fraction of the community agrees with that, there is manifoldly cited work that finds low-mass Pop III progenitors, and the authors should not let that unmentioned, but also cite eg Stacy+16, Greif+11 and Clark+11.
\end{referee}
-I will look at and add these citations and detail to the introduction.

\begin{referee}
The authors name a few processes for suppressing Pop III star formation. For completeness, they should also mention eg X-ray heating (citing eg Jeon+14).
\end{referee}
-I will add this detail (https://arxiv.org/pdf/1310.7944.pdf)
-Jeon+14 seems to show that X-ray heating does not affect star formation, but 
the clumping factor is reduced, and thus makes it easier to keep the reionized 
gas ionized. 

\begin{referee}
The authors cite the well-known study from Machacek+01. However, a more recent result would be from the study by Visbal+14, giving a lower mass limit. Since the authors find a lower mass limit than Machacek+01, it would be interesting to see how this compares to the lower mass limit from the more modern paper.
\end{referee}
-Visbal+14 provides an equation for Mcrit (eqn. 4), I have plotted this previously. I have plotted this for z=20, and adjusted the trenti line for that redshift as well. May have to adjust the plot more. I will add detail about the line to the paper. (https://arxiv.org/pdf/1402.0882.pdf) 

\begin{referee}
Methods:

In Wise+12, three different environment regions are picked. Please provide more description on the rare peak simulation here and discuss why this one is appropriate in this context.
\end{referee}
-RP stands for radiation pressure in Wise+12, not rare peak. The simulation is the same size as ours, and contains similar physics, including the radiation pressure and soft UV background, which is why we reference it here. The differences are the inclusion of H2 self-shielding and updated cosmo parameters.
 
\begin{referee}
The initialization at z=130 is very late. I am aware that it is impossible to repeat this simulation starting at an earlier redshift, but would advise for future work.
\end{referee}

In hindsight, we agree that we initialized the simulation too late, even with 2nd order Lagrangian perturbation theory.  The maximum baryon overdensity is 1.64$\bar{\rho}_{\rm b}$, and the maximum DM particle displacement is 2.5 $\times$ (cell width).  In future studies, we will start at earlier times when the evolution is solidly in the linear regime.  We thank the referee for pointing this out.

\begin{referee}
Why do the authors pick a non-zero metalicity threshold for Pop III star formation? How many Pop III stars form between Z=0 and Z=5e-6 Z\_sol?
\end{referee}

We chose this metallicity threshold based on the work of Omukai et al. (2005), Schneider et al. (2006), who found that dust cooling was efficient enough above this metallicity to induce low-mass star formation. About 10\% of Pop III stars form with metallicities between zero and $5 \times 10^{-6} Z_\odot$.  Stars only obtain this extremely low (non-zero) metallicity when metal mixing is incomplete.  This usually occurs when a blastwave is mixing into a nearby halo, and the core collapses before it can fully mix.  Because of the randomness of the IMF sampling and errors from numerical diffusion and other solvers, we do not place great emphasis on stars in this metallicity range.  It is unclear what would be the actual stellar metallicity when it reaches main sequence, given our 0.1 proper pc resolution.  We have made a note of this in Section 2.2.1.

\begin{referee}
What is the physical motivation for a density threshold of 1.e6/cc? Is there convergence?
\end{referee}

Our choice was motivated by both previous work and our refinement strategy rather than convergence. Previous work (e.g. O'Shea et al. 2007; Yoshida et al. 2007) and later groups (e.g. Hirano et al. 2015) have shown that the number density is $\sim 10^6 \cubecm$ at our resolution limit of $0.09$ proper pc at $z=10$.  Secondly, we chose this threshold to be similar to the density that would trigger another level of AMR past our maximum level of 12 (dx = 0.09 proper pc at $z=10$).  At these scales, the grid is primarily refined from the Jeans length criterion, which is resolved by at least 8 cells.  Solving for the density in the Jeans length equation, given a typical temperature $T=300$~K with primordial cooling and a Jeans Length of $8 \times \textrm{dx}$, we obtain a number density $n = 5.7 \times 10^4 \cubecm$.  This number density does not exactly match the chosen density criterion because some oversight when setting up the simulation.  In the case where the density SF criterion is larger than $n$, the gas will continue to condense on a faster timescale than its surroundings.  Thus we are confident that this slight mismatch does not affect our results.

We have added the AMR refinement criteria to Section 2.1 and our motivation to this threshold to Section 2.2.1.

\begin{referee}
Similarly, why do the authors chose a H2 fraction of 1.e-3? (Is is a mass fraction or a number fraction?) How many Pop III stars do not form because of this threshold?
\end{referee}

We chose this critical H$_2$ fraction to match with the values found in previous work (e.g. Yoshida et al. 2007; Omukai et al. 2010) at the density threshold of $10^6 \cubecm$.  It is a number fraction.  We have added to Section 2.2.1 to reflect this detail.

In our original formulation (Abel et al. 2007), we implemented this restriction to avoid spurious star particle formation in D-type ionization fronts that can meet all of the other criteria.  But in the presence of an extremely strong Lyman-Werner flux from a nearby ($\sim$1 pc) star, star formation is unlikely to occur in the shell because it is irradiated and not self-gravitating. The last question is an interesting question, but it is difficult to determine because these conditions (high density but low H$_2$ fractions) are short-lived just after star formation.  Thus we cannot give a definite answer to it, but from the original formulation, we would estimate that the number of Pop III stars would increase by a factor of several with multiple cells in the quasi-spherical ionization front triggering star particle formation.  Because of these uncertainties, we have not added any of these details to the manuscript.

\begin{referee}
Do the authors set fixed mass limits to the IMF and if yes, which are they?  If not, which is the highest / lowest mass they find in their random sampling?  And which fraction of the Pop III stars in the simulation end up as type II SNe / BHs / PISN?
\end{referee}
-We do set fixed mass limits, between 1 and 300 Msol. I added this detail. 

-Plot IMF of P3 stars, find the fractional between each mass range. Watch units, will have to divide by M. Need to figure out how to plot this still.

\begin{referee}
One effect than can weaken the shielding is shifting out of the Lyman-Werner resonance lines (see Hartwig+15). I am wondering how much this would affect the study presented here.
\end{referee}
-Will have to look at Hartwig+15. 

\begin{referee}
In Figure 1, please mention in the caption or at the top / bottom of the left axis that a shielding factor of 1 corresponds to no shielding and a shielding factor of 0 corresponds to high shielding, as these nomenclatures are not immediately understandable for the reader.
\end{referee}
-This detail has been added. 

\begin{referee}
The analytic calculation of the shielding factor is a nice property of the paper. However, assuming a constant H2 fraction seems not appropriate. Please repeat this exercise for a more realistic halo profile, which could be found in the simulations or in the literature, eg by Machacek+01 themselves.
\end{referee}
-I have pulled data from O'Shea and Norman 07 from their Figure 14, where there is a LW background. 

\begin{referee}
Results:

Figure 2 is quite empty, it would be useful to include the halo mass function of the earliest Pop III formation redshift with its analytic prediction. In addition, the authors could provide a line or an arrow to show their halo mass resolution limit.
\end{referee}
-I have the data for the halo masses from the earliest dataset where star formation ahppens, but it does not line up well with the analytic formula....

-Added a vertical line at $10^5 M_{\odot}$. Detail also added to caption.
2-Tried plotting the HMF of the earliest Pop III formation redshift, but I need to collect the HMF of that dataset -- having troubles remembering how this was produced 


\begin{referee}
Figure 3 looks very nice. It would be helpful if an additional line that only 
provides the global LWBG (in units of J21) was included.
\end{referee}
-Woohoo! Figure 3 does look nice!
-I have added the background LW. I will add the detail that the JLW that is plotted for the whole plot is coming from only sources. 

\begin{referee}
The authors find no correlation between the host halo mass and the Lyman-Werner background intensity. However, they only look at the LWBG flux at the time of Pop III formation - is there maybe a correlation between the time integrated LW flux?
\end{referee}
This would be difficult to calculate. It would require constant tracking of halos through time to calculate the time integrated LW flux. 

\begin{referee}
Figures 7 and 8 show very nice histograms as an additional panel on the right.  I suggest to include a histogram on the spread of the creation time in Figure 9 similar to Figures 7 and 8.
\end{referee}
-Added the histogram to Figure 9. 

\begin{referee}
Discussion:

The authors give three reasons for the halo mass spread. Can they estimate which of the three processes (BH formation, dynamical heating and temporal LWBG fluctuations) is responsible for the spread by which fraction?
\end{referee}

This is an interesting question, but it is outside the scope of this paper.  It most likely depends on the large-scale environment, halo mass accretion history, and halo star formation history, but how these factors affect each other is unclear.  In practice, it would be difficult to determine the relative importance of dynamical heating and temporal LWBG fluctuations because of the time variations in the heating rate and suppression of H$_2$ cooling.  Because of these reasons, we do not give any estimates in the discussion.

The occurrence of BH formation and no metal enrichment depends on the chosen Pop III IMF, and we have addressed in an earlier answer and noted this fraction in the discussion.

\begin{referee}
One of the most extensive studies on over 1000 minihaloes was performed by Hirano+15, who also include a Lyman-Werner background. Their work should be put in comparison as well.
\end{referee}
Hirano+15 is looking at the distribution of stellar masses to calculate the IMF of p3 stars, which is why we quote them in the introduction. 
-Their Fig. 3 shows the number of halos versus virial mass that host primordial star-forming clouds. 
-Using their eqn. 6 they find Mvir = $2.1 \times 10^5$ at z=30, and Mvir = $9.9 \times 10^5$ at z=10, which they say is in good agreement with their Fig. 3. 
(https://arxiv.org/pdf/1501.01630.pdf)

\begin{referee}
In Figure 10, the result from Yoshida+03 is hard to see, maybe use a different symbol / colour.
\end{referee}
-Adjusted. Changed the symbol to a pentagon and the outline color black. Okay?

\begin{referee}
Since the authors study star formation in minihaloes, it would be appropriate to also mention Kitayama+04 and Schauer+15 for the Lyman-Werner escape fractions from minihaloes.
\end{referee}
-Kitayama+04: HII regions are governed by an initial slow and weak D-type ionization front followed by a rapid R-type front. For there to be complete ionization, the transition between the types is critical. They find that in small mass halos, less than $10^6 M_{\odot}$, this critical transition takes place fairly rapidly, allowing for high escape fractions of greater than 80\%. In larger mass halos, the ionization front remains as a D-type, and the escape fraction is essentially zero. They use a simulation to study these HII regions in similar mass halos as ours and within a similar redshift range. Their Fig. 4 shows the ionizing and LW escape fractions as a function of halo mass for different stellar masses. (https://arxiv.org/pdf/astro-ph/0406280.pdf)

-Schauer+15: They run simulations of LW escape fractions from P3 stars in $10^5 - 10^7 M_{\odot}$ halos. They find escape fractions from 0-85\%, and that H2 shielding by neutral hydrogen also decreases escape fractions compared to escape fractions predicted by only H2 self-shielding. Fig.3 shows escape fractions for different models and different P3 stellar masses. When H2 shielding by neutral hydrogen is not included, escape fractions are overpredicted. (https://arxiv.org/pdf/1506.04796.pdf)

\begin{referee}
The authors mention a few times that streaming velocities are another pathway of suppressing Pop III formation (also in the introduction). While they cite some older work, they however fail to compare to more recent work eg by Hirano+18 or Schauer+19 (who do perform a systematic study not only based on a few, single haloes).
\end{referee}
-Hirano+18: Investigate the collapse of primordial gas to form P3 stars when the energy transfer between baryon and dark matter fluids (due to non-gravitational scattering) is included.

\end{document}
