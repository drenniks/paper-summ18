% paper.tex
%
% LaTeX template for creating an MNRAS paper
%
% v3.0 released 14 May 2015
% (version numbers match those of mnras.cls)
%
% Copyright (C) Royal Astronomical Society 2015
% Authors:
% Keith T. Smith (Royal Astronomical Society)

% Change log
%
% v3.0 May 2015
%    Renamed to match the new package name
%    Version number matches mnras.cls
%    A few minor tweaks to wording
% v1.0 September 2013
%    Beta testing only - never publicly released
%    First version: a simple (ish) template for creating an MNRAS paper

%%%%%%%%%%%%%%%%%%%%%%%%%%%%%%%%%%%%%%%%%%%%%%%%%%
% Basic setup. Most papers should leave these options alone.
\documentclass[a4paper,fleqn,usenatbib]{mnras}

% MNRAS is set in Times font. If you don't have this installed (most LaTeX
% installations will be fine) or prefer the old Computer Modern fonts, comment
% out the following line
\usepackage{newtxtext,newtxmath}
% Depending on your LaTeX fonts installation, you might get better results with 
% one of these:
%\usepackage{mathptmx}
%\usepackage{txfonts}

% Use vector fonts, so it zooms properly in on-screen viewing software
% Don't change these lines unless you know what you are doing
\usepackage[T1]{fontenc}
\usepackage{ae,aecompl}


%%%%% AUTHORS - PLACE YOUR OWN PACKAGES HERE %%%%%

% Only include extra packages if you really need them. Common packages are:
\usepackage{graphicx}	% Including figure files
\usepackage{amsmath}	% Advanced maths commands
\usepackage{amssymb}	% Extra maths symbols
\usepackage{mathtools}
\usepackage{hyperref}
%\usepackage{subcaption}
%\usepackage{caption}
%%%%%%%%%%%%%%%%%%%%%%%%%%%%%%%%%%%%%%%%%%%%%%%%%%

%%%%% AUTHORS - PLACE YOUR OWN COMMANDS HERE %%%%%

\newcommand{\fesc}{\ifmmode{f_{\rm esc}}\else{$f_{\rm esc}$}\fi}
\newcommand{\fescs}{\ifmmode{f_{\rm esc}^\star}\else{$f_{\rm esc}^\star$}\fi}
\newcommand{\kms}{\ifmmode{{\;\rm km~s^{-1}}}\else{km~s$^{-1}$}\fi}
\newcommand{\fgas}{\ifmmode{{f_{\rm gas}}}\else{$f_{\rm gas}$}\fi}
\newcommand{\cubecm}{\ifmmode{{\rm cm^{-3}}}\else{cm$^{-3}$}\fi}
\newcommand{\ztwo}{\ifmmode{{\rm [Z_2/H]}}\else{[Z$_2$/H]}\fi}
\newcommand{\zthree}{\ifmmode{{\rm [Z_3/H]}}\else{[Z$_3$/H]}\fi}
\newcommand{\lsim}{\lower0.3em\hbox{$\,\buildrel <\over\sim\,$}}
\newcommand{\gsim}{\lower0.3em\hbox{$\,\buildrel >\over\sim\,$}}
\newcommand{\li}{\noindent$\bullet$\quad}
\newcommand{\lli}{$\bullet$\quad}
\newcommand{\lcdm}{$\Lambda$CDM}
\newcommand{\flux}{erg s$^{-1}$ cm$^{-2}$ Hz$^{-1}$}
\newcommand{\emis}{erg s$^{-1}$ cm$^{-2}$ Hz$^{-1}$ sr$^{-1}$}
\newcommand{\sfr}{\ifmmode{\textrm{M}_\odot \,\textrm{yr}^{-1} \,\textrm{Mpc}^{-3}}\else{M$_\odot$ yr$^{-1}$ Mpc$^{-3}$}\fi}
\newcommand{\hsfr}{\ifmmode{\textrm{M}_\odot\, \textrm{yr}^{-1}}\else{M$_\odot$ yr$^{-1}$}\fi}
\newcommand{\ssfr}{Gyr$^{-1}$}
\newcommand{\arp}{$a_{\rm rp}$}
\newcommand{\agrav}{$a_{\rm grav}$}
\newcommand{\eavg}{\ifmmode{\langle E_\gamma \rangle}\else{$\langle E_\gamma \rangle$}\fi}
\newcommand{\hst}{{\it HST}}
\newcommand{\enzo}{{\sc enzo}}
\newcommand{\yt}{{\sc yt}}
\newcommand{\moray}{{\sc enzo+moray}}
\newcommand{\Ms}{\ifmmode{M_\odot}\else{$M_\odot$}\fi}
\newcommand{\vrms}{\ifmmode{v_{\rm rms}}\else{$v_{\rm rms}$}\fi}
\newcommand{\mmin}{M$_{min}$}
\newcommand{\hh}{H$_2$}
\newcommand{\Ol}{$\Omega_\Lambda$}
\newcommand{\Om}{$\Omega_M$}
\newcommand{\Ob}{$\Omega_b$}
\newcommand{\theat}{$t_{\rm{heat}}$}
\newcommand{\tcool}{$t_{\rm{cool}}$}
\newcommand{\rcool}{$r_{\rm{cool}}$}
\newcommand{\tcross}{$t_{\rm{cross}}$}
\newcommand{\tdyn}{$t_{\rm{dyn}}$}
\newcommand{\tkh}{$t_{\rm{KH}}$}
\newcommand{\tH}{$t_{\rm{H}}$}
\newcommand{\tvir}{\ifmmode{T_{\rm{vir}}}\else{$T_{\rm{vir}}$}\fi}
\newcommand{\mvir}{\ifmmode{M_{\rm{vir}}}\else{$M_{\rm{vir}}$}\fi}
\newcommand{\rvir}{\ifmmode{r_{\rm{vir}}}\else{$r_{\rm{vir}}$}\fi}
\newcommand{\rr}{$r_{200}$}
\newcommand{\lya}{Ly$\alpha$}
\newcommand{\jj}{\ifmmode{J_{21}}\else{$J_{21}$}\fi}
\newcommand{\flw}{\ifmmode{F_{LW}}\else{$F_{LW}$}\fi}
\newcommand{\kph}{\ifmmode{k_{\rm ph}}\else{$k_{\rm ph}$}\fi}
\newcommand{\tv}{$\langle T \rangle_{\rm v}$}
\newcommand{\tm}{$\langle T \rangle_{\rm m}$}
\newcommand{\msun}{{\rm\,M_\odot}} 
\newcommand{\lsun}{{\rm\,L_\odot}}
\newcommand{\zsun}{\ifmmode{\rm\,Z_\odot}\else{$\rm\,Z_\odot$}\fi}
\newcommand{\etal}{et al.\ }
\newcommand\tento[1]{$10^{#1}$}
\newcommand\step[1]{\textit{Step #1}.--}
\newcommand\halo[1]{\textit{Halo #1}.--}
\newcommand{\hi}{H {\sc i}}
\newcommand{\hii}{H {\sc ii}}
\newcommand{\hei}{He {\sc i}}
\newcommand{\heii}{He {\sc ii}}
\newcommand{\heiii}{He {\sc iii}}
\newcommand{\nhi}{\ifmmode{N_{\rm HI}}\else{$N_{\rm HI}$}\fi}
\newcommand\unit[1]{\; \textrm{#1}}
\newcommand{\pem}{\unit{s}^{-1} \unit{cMpc}^{-3}}
\newcommand{\music}{{\sc Music}}
\newcommand{\rockstar}{{\sc Rockstar}}

\newcommand{\refs}{(\textbf{Add references})}
\newcommand{\jhw}[1]{{\color{red} \bf (JHW: #1)}}


%%%%%%%%%%%%%%%%%%%%%%%%%%%%%%%%%%%%%%%%%%%%%%%%%%
%% Additions from AASTeX
%%%%%%%%%%%%%%%%%%%%%%%%%%%%%%%%%%%%%%%%%%%%%%%%%%

%\newcommand\ion[2]{#1$\;${\small\rmfamily\@Roman{#2}}\relax}
\def\eps@scaling{1.0}% 
\newcommand\epsscale[1]{\gdef\eps@scaling{#1}}% 
\newcommand\plotone[1]{% 
 \centering 
 \leavevmode 
 \includegraphics[width={\eps@scaling\columnwidth}]{#1}% 
}% 
\newcommand\plottwo[2]{% 
 \centering 
 %\leavevmode 
 %\columnwidth=.5\columnwidth 
 \includegraphics[width={\eps@scaling\columnwidth}]{#1}% 
 \hfil 
 \includegraphics[width={\eps@scaling\columnwidth}]{#2}% 
}% 

% \newcommand{\nat}{Nature}
% \newcommand{\aj}{AJ}
% \newcommand{\apj}{ApJ}
% \newcommand{\apjl}{ApJL}
% \newcommand{\apjs}{ApJS}
% \newcommand{\mnras}{MNRAS}
% \newcommand{\aap}{A\&A}
% \newcommand{\pasj}{PASJ}
% \newcommand{\pasp}{PASP}
% \newcommand{\araa}{ARA\&A}
% \newcommand{\bain}{Bull.~Astron.~Inst.~Netherlands}
% \newcommand{\memsai}{Mem.~Soc.~Astron.~Italiana}
% \newcommand{\physrep}{Physics Reports}
% \newcommand{\prd}{Phys.~Rev.~D}
% \newcommand{\jcap}{JCAP}

\hyphenation{sSFR sSFRs}

\makeatletter
\newcommand*{\rom}[1]{\expandafter\@slowromancap\romannumeral #1@}
\makeatother

% Please keep new commands to a minimum, and use \newcommand not \def to avoid
% overwriting existing commands. Example:
%\newcommand{\pcm}{\,cm$^{-2}$}	% per cm-squared

%%%%%%%%%%%%%%%%%%%%%%%%%%%%%%%%%%%%%%%%%%%%%%%%%%

%%%%%%%%%%%%%%%%%%% TITLE PAGE %%%%%%%%%%%%%%%%%%%

% Title of the paper, and the short title which is used in the headers.
% Keep the title short and informative.
\title[Cradles of the first stars]{Cradles of the first stars: self-shielding, halo masses, and multiplicity}

% The list of authors, and the short list which is used in the headers.
% If you need two or more lines of authors, add an extra line using \newauthor
\author[Danielle Skinner et al.]{
Danielle Skinner,$^{1}$\thanks{E-mail: drenniks@gatech.edu}
John H. Wise,$^{1}$
\\
% List of institutions
$^{1}$Center for Relativistic Astrophysics, Georgia Institute of Technology, 
Atlanta, GA 30332, USA\\
}

% These dates will be filled out by the publisher
%\date{Accepted XXX. Received YYY; in original form ZZZ}

% Enter the current year, for the copyright statements etc.
\pubyear{2018}

% Don't change these lines
\begin{document}
\label{firstpage}
\pagerange{\pageref{firstpage}--\pageref{lastpage}}
\maketitle

% Abstract of the paper
\begin{abstract}
The formation of Population III (Pop III) stars is a critical step in the evolution of the early universe. To understand how these stars affected their metal-enriched descendants, the details of how, why and where Pop III formation takes place needs to be determined. One of the processes that is assumed to greatly affect the formation of Pop III stars is the presence of a Lyman-Werner (LW) radiation background, that destroys \hh{}, a necessary coolant in the creation of Pop III stars. Self-shielding can alleviate the effect the LW background has on the \hh{} within halos. In this work, we perform a cosmological simulation to study the birthplaces of Pop III stars, using the adaptive mesh refinement code \textsc{Enzo}. We investigate the distribution of host halo masses and its relationship to the LW background intensity. Compared to previous work, halos form Pop III stars at much lower masses, up to a factor of a few, due to the inclusion of \hh{} self-shielding. We see no relationship between the LW intensity and host halo mass. Most halos form multiple Pop III stars, with a median number of four, up to a maximum of 16, at the instance of Pop III formation. Our results suggest that Pop III star formation may be less affected by LW radiation feedback than previously thought and that Pop III multiple systems are common. 
\end{abstract}{}

% Select between one and six entries from the list of approved keywords.
% Don't make up new ones.
\begin{keywords}
star formation -- methods:numerical -- cosmology
\end{keywords}

%%%%%%%%%%%%%%%%%%%%%%%%%%%%%%%%%%%%%%%%%%%%%%%%%%

%%%%%%%%%%%%%%%%% BODY OF PAPER %%%%%%%%%%%%%%%%%%
%====================================================================
\section{Introduction}
%====================================================================

The first generation of stars transformed a cold and dark universe
by illuminating and enriching up their neighborhoods with radiation
and heavy elements. The formation of these first stars is a crucial step in the cosmological evolution of the universe because the metals and feedback they deliver to their local environments are necessary for further star production and chemical enrichment. Without these initial stars, heavier metals would not have been produced, and a different universe would be observed than what is observed today. These stars are, by definition, metal-free (Population III; Pop III) and are thought to be generally massive \citep{ABN02, Bromm02_P3, Turk09, Hosokawa11, Hosokawa16, Hirano15}. A substantial fraction of these stars will generate prodigious amounts of ionizing photons and will end their lives in some form of a supernova \citep[e.g.][]{Schaerer02, Heger02}. The supernova will spread the enriched guts of the Pop III star out across its local environment, providing the area with elements the universe has not yet seen. 

Once a halo becomes chemically enriched by the death of its Pop III stars, then by definition, it can no longer form more Pop III stars. This marks the end of metal-free star formation in that pregalactic object. Understanding the mixing of metals into the environment of these halos is necessary to constrain the reach of these metals and the effects they have for future star formation. Numerical studies have shown that turbulence within halos can mix metals well down to their resolution limit \citep{Wise08_Gal, Greif10, Smith15}. Stars will continue to form in halos, but with an increased metal abundance, these stars are now labeled as metal-enriched stars. These stars still are metal-poor, having metallicities of 10$^{-6}$ to 10$^{-2}$ of the solar abundance \citep{Chiaki16, Chiaki18, Ritter16}, have a direct chemical connection to Pop III stars, and survive until the present day \citep{Gnedin06, Tumlinson10, Griffen18, Magg18, Ezzeddine19}. Understanding the formation and chemical abundance of second generation stars can provide more evidence and insight for the earlier generation of Pop III stars.

Without metals and dust to facilitate efficient radiative cooling, Pop III stars rely on \hh{} formation to cool the gas. These molecules are however fragile to dissociation from Lyman-Werner (LW) radiation in the energy range 11.2--13.6~eV, through the Solomon process \citep{Field66, Stecher67}. This is a two-step process through which H$_{2}$ is excited to a higher state, H$_{2}^{\ast}$, via absorption of a LW photon:
\begin{equation} \label{Solomon1}
	H_{2} + \gamma \rightarrow  H_{2}^{\ast}
\end{equation}
This excited state then has a probability of dissociating into two hydrogen atoms:
\begin{equation} \label{Solomon2}
	H_{2}^{\ast} \rightarrow H + H
\end{equation}

Diffuse gas is optically thin to LW radiation, thus a background builds over time and can suppress \hh{} formation, delaying Pop III star formation. Furthermore, nearby sources of LW radiation can boost the intensity above the background value which facilitates further \hh{} dissociation in minihalos. This process can be counteracted with a sufficient amount of \hh{} already present within a halo, via \hh{} self-shielding. Halos with an \hh{} column density of N$_{\rm H2}$ $\geq$ 10$^{14}$ cm$^{-2}$ can suppress the photodissociation of \hh{} by LW photons \citep{Draine96}, giving the halo a chance to form Pop III stars. Other sources of the suppression of Pop III star formation are streaming baryonic velocities \citep{Tselia11, Greif11_Delay, Naoz12,OLeary12}, arising from recombination, and dynamical heating, occurring during the virialization of halos \citep{Yoshida03, Fernandez14}. 

Given the paradigm of hierarchical structure formation, halos grow through smooth accretion and successive mergers of smaller halos. But in which halos do these first generations of stars form? Their formation rates and locations are important to constrain because they influence the very beginnings of galaxy formation and cosmic reionization. Various semi-analytic investigations have been conducted to learn more about the halo collapse criterion, and thus, which halos can host Pop III stars. \citet{Tegmark97} discovered that halos can have a strongly redshift dependent, minimum baryonic mass of 10$^{6}$ M$_{\odot}$ at $z \approx$ 15. In particular, they derive an analytic expression for the fraction of \hh{} needed in a halo for efficient cooling, and determine which  halos can cool in a Hubble time. \citet{Visbal18} devised a semi-analytic model for the formation of Pop III stars and the transition to metal-enriched stars. They find that varying the Pop III star formation efficiency, the time from a Pop III supernova to metal-enriched star formation, the external enrichment of the halo, and the ionizing escape fraction, leads to large differences in the star formation history of Pop III and metal-enriched stars. This method is useful for exploring the wide parameter space of star formation in the early universe, and could lead to further constraints in the future. 

Previous work have mainly investigated and established the lower limit for a halo to host Pop III stars, or for a halo to collapse. However, subsequent simulations have shown that they do not necessarily form at this minimum due to the aforementioned physical processes playing a role. \citet{Mebane18} used a semi-analytic model of early star formation and found that the LW background coming from the rapidly increasing supply of Pop III stars becomes responsible for suppressing Pop III star formation. \citet{Griffen18} also find that the LW background can significantly suppress the amount of potential sites for Pop III star formation. 

Numerical simulations have also been employed to investigate collapse thresholds and Pop III star formation. \citet[hereafter M01]{Machacek01} found similar results to those mentioned previously, in that the LW feedback can suppress the collapse of small mass halos. They fit a simple analytic expression for the mass threshold of halos given a particular LW flux:
\begin{equation} \label{mthresh}
	M_{\rm TH} ( M_{\odot} ) = 1.25 \times 10^{5} + 8.7  \times 10^{5} \left( \frac{4 \pi J_{\rm LW}}{10^{-21}} \right)^{0.47} .
\end{equation}
We will compare our results with this relation in future sections. \citet{Yoshida03} also finds that cooling is inefficient in halos with a LW background > 0.01 J$_{\rm 21}$, although with sufficient \hh{} shielding, halos are able to cool in the given LW background radiation. In fact, \citet{Wise07_UVB} found that central shocks drive \hh{} formation, allowing the halo to cool in a LW background intensity of 1 J$_{21}$. They find that \hh{} cooling is always a dominant process, even in large LW background fluxes. \citet{OShea08} also find that Pop III star formation can occur in relatively high LW backgrounds, implying that the LW background may not be a complete indicator of whether or not Pop III stars will form in a given halo.    

In this paper, we focus on the distribution of host halo masses, not just the minimum, and its dependence on redshift and the LW background, augmenting results from prior work. We aim to provide further insight into the host halos of Pop III stars. In \S 2 the methods of the simulation, implementation of star formation, feedback, and \hh{} self-shielding are described. In \S 3 we present the results of our analysis. In \S 4 we discuss the results in more detail and compare with previous work. In \S 5 we conclude our discussion with a summary of our results and the implications therein.

%====================================================================
\section{Methods}
%====================================================================
\subsection{Simulation setup}
%====================================================================
We run and analyze a cosmological simulation with the adaptive mesh refinement (AMR) code \textsc{Enzo} \citep{Enzo} and the toolkit \yt{} \citep{yt_full_paper}. \textsc{Enzo} uses an N-body adaptive particle-mesh solver \citep{Efstathiou85, Couchman91} to follow the dark matter (DM) dynamics. We use the nine-species (\hi, \hii, \hei, \heii, \heiii, e$^{-}$, H$_{2}$, H$_{2}^{+}$, H$^{-}$) non-equilibrium chemistry model \citep{Abel97, Anninos97}. This simulation is similar to the RP simulation in \citet[hereafter W12]{Wise12_RP}, but with updated cosmological and Pop III parameters, and the inclusion of \hh{} shielding.

We simulate a 1 Mpc$^{3}$ comoving box with a 256$^{3}$ base grid 
resolution and a dark matter paticle mass of 2000.54 M$_{\odot}$. It has a maximum refinement level of 12 which provides a maximal comoving resolution of $\sim$1 pc. The simulation is 
initialized with \music{} \citep{Hahn11_MUSIC} at $z$ = 130 and uses the cosmological parameters from the Planck collaboration best fit 
\citet{Planck13_Cosmo}: $\Omega_{\rm M}$ = 0.3175, $\Omega_{\Lambda}$ = 
0.6825, $\Omega_{\rm DM}$ = 0.2685, and h = 0.6711, with the variables 
having their usual definitions. The simulation is run until $z$ = 9.32, when it becomes too computationally expensive to continue. At this point, about 20\% of the volume is over 10\% ionized. We output 918 datasets,  roughly 0.5 Myr apart. In this paper, we focus only on outputs where Pop III stars have just formed, from $z$ = 27.23 to 9.39. 

A time-dependent LW optically thin radiation background is applied in the simulation. This was fit in W12 (see their Eq. 16) and is consistent with the values in \citet{Trenti09_SFR}. The background evolution of the specific intensity takes the following form:
\begin{equation} \label{LWbg}
	\log_{10}(J_{\rm LW}/J_{21}) = A + Bz - Cz^{2} + Dz^{3} - Ez^{4}
\end{equation}
where $(A, B, C, D, E)$ = $(-2.567, 0.4562, - 0.02680, 5.882 \times 10^{-4}, - 5.056 \times 10^{-6})$ and J$_{21}$ is a specific intensity of 10$^{-21}$ erg s$^{-1}$ cm$^{-2}$ Hz$^{-1}$ sr$^{-1}$.  This form gives a maximum of $J_{\rm LW}/J_{21} = 0.85$ at a redshift of 13.5 and quickly and continually declines for higher redshifts. We use adaptive ray tracing \citep{Abel02_RT, Wise11_Moray} to evolve the ionizing radiation field. Centered on each metal-enriched and Pop III star particle, we model the \hh{} dissociating radiation field with an optically thin, inverse square profile. These LW sources are added on top of the background intensity described above.
%====================================================================
\subsection{Star formation}
%====================================================================
In this section, we briefly discuss our implementation of Pop III and metal-enriched star formation. We do not consider the formation and feedback from stellar remnants or asymptotic giant branch (AGB) stars. For further details, we refer the reader to W12. 

%====================================================================
\subsubsection{Pop III Star Formation and Feedback }
%====================================================================
The utilized Pop III star formation model is the same as in \citet{Wise08_Gal} but with updated parameters. Each star particle represents a single massive star, and is formed in a cell when the following criteria are met: 
\begin{enumerate}
	\item a metallicity Z $\leq$ 5 $\times$ 10$^{-6}$ Z$_{\odot}$

	\item a gas number density n > 10$^{6}$ cm$^{-3}$

	\item converging gas flow, $\nabla \cdot \mathbf{v_{gas}}$ < 0 

	\item molecular hydrogen fraction, f$_{\rm H2}$ > 10$^{-3}$
\end{enumerate}

If within 1 pc, multiple cells meet this criteria, then a single Pop III star forms at the center of mass of these cells. The mass of the Pop III star is randomly sampled from an initial mass function (IMF) of the form:
\begin{equation} \label{IMF}
	f(\log M)dM = M^{-1.3} \exp \left[-\left( \frac{M_{\rm char}}{M}\right)^{1.6} \right]dM
\end{equation}
where M$_{\rm char}$ = 20 M$_{\odot}$ \citep{Hirano17}. This characteristic mass is different from the W12 choice of 100 M$_{\odot}$. This results in an exponential cutoff before M$_{\rm char}$ and a power-law IMF after. Once the cell meets the above criteria and the stellar mass is chosen, an equal amount of gas is removed from the grid in a sphere which contains twice the stellar mass that is centered on the new star particle. The new Pop III star then gains the mass-weighted velocity of the gas contained within that sphere. In this simulation, we do not track any small scale fragmentation which may form low mass stars \citep{Greif11_P3Cluster}. We are only tracking massive Pop III star formation.

We use the mass-dependent hydrogen ionizing and LW luminosities and lifetimes from \citet{Schaerer02} and a mass-independent photon energy of E$_{\rm ph}$ = 29.6 eV, which is appropriate for the nearly mass-independent surface temperature of Pop III stars, at 10$^{5}$ K. A Type II supernova results if the Pop III star dies with a mass between 11 $\leq$ M$_{\star}$/M$_{\odot}$ $\leq$ 40. A pair instability supernova (PISNe) results if the star dies with a mass between 140 $\leq$ M$_{\star}$/M$_{\odot}$ $\leq$ 260. Outside of these mass ranges, the Pop III star dies as a black hole. In this simulation, no black hole physics are implemented, so the Pop III star simply turns into a collisionless particle. When 11 $\leq$ M$_{\star}$/M$_{\odot}$ < 20, the star dies as a normal Type II supernova with an energy of 10$^{51}$ erg. From 20 $\leq$ M$_{\star}$/M$_{\odot}$ $\leq$ 40, the star dies as a hypernova with energies taken from \citet{Nomoto06}. The PISNe have explosion energies from \citet{2002ApJ...567..532H}, of which the analytic function fit to their models can be seen in Eq. 12 of W12. 

%====================================================================
\subsubsection{Pop II  Star Formation and Feedback}
%====================================================================
The Pop II star formation model is the same as \citet{Wise09}, which is equivalent to the above criteria for Pop III star formation but without the molecular hydrogen fraction requirement, and with a metallicity greater than the above value. Star formation is only allowed for gas with temperatures $T$ < 1000 K, to ensure that the volume is cold and collapsing. Contrary to Pop III stars representing single star particles, metal-enriched star particles are modelled as clusters of stars. We enforce a minimum mass of M$_{\rm min}$ = 1000 M$_{\odot}$ for these star particles. 

The lifetime of these particles is 20 Myr, during which they radiate 6000 hydrogen ionising photons per stellar baryon, or 1.13 $\times$ 10$^{46}$ photons s$^{-1}$ M$_{\odot}^{-1}$, which is appropriate for a Salpeter IMF \citep{Schaerer03}. Their spectra are approximated with a monochromatic spectrum with an energy of 21.6 eV at a constant luminosity. After living for 4 Myr, the stars begin to return supernova energies of 6.8 $\times$ 10$^{48}$ erg s$^{-1}$ M$_{\odot}^{-1}$ back to their surroundings.

%====================================================================
\subsection{\hh{} Self-Shielding}
%====================================================================
We use the Sobolev-like approximation from \citet{Wolcott11} to model \hh{} self-shielding. To determine the \hh{} shielding factor, the column density of \hh{} needs to be calculated:
\begin{equation} \label{Nh2}
	N_{\rm H2} = n_{\rm H2}L_{\rm char} \ ,
\end{equation}
where n$_{\rm H2}$ is the number density of \hh{} and L$_{\rm char}$ is a characteristic length over which n$_{\rm H2}$ is assumed to be constant. The method employed in this simulation is to define the characteristic length as:
\begin{equation} \label{Lchar}
	L_{\rm char} = \frac{\rho}{|\nabla \rho|} \ . 
\end{equation}
This Sobolev-like method determines the length over which gas with density $\rho$ is diminished. They found that the most accurate, non-ray tracing, method is to use a single Sobolev-like length determined from the mean Sobolev-like length over all Cartesian directions. 

Upon numerically calculating the shielding factor of \hh{} for simulated protogalaxies, Wolcott-Green et al. determined that a slight adjustment to the shielding factor from \citet{Draine96} is sufficient to account for inaccuracies at higher temperatures. The implemented shielding factor is then (see Eq. 10 in \citet{Wolcott11}):
\begin{equation} \label{shield}
	\begin{multlined}
	f_{\rm sh}(N_{\rm H2}, T) = \frac{0.965}{(1+x/b_{5})^{1.1}} + \frac{0.035}{(1+x)^{0.5}}  \times \\ \exp [-8.5 \times 10^{-4} (1+x)^{0.5}]
	\end{multlined}
\end{equation}
Here, x $\equiv$ N$_{\rm H2}$/5 $\times$ 10$^{14}$ cm$^{-2}$, b$_{5}$ $\equiv$ b/10$^{5}$ cm s$^{-1}$, and $b$ is the Doppler broadening parameter. This shielding factor acts as a multiplier to the \hh{} dissociation rate, where if the shielding factor equals one, there is no shielding, and zero if there is maximum shielding.

%====================================================================
\subsection{Analytic Calculation of the Shielding Factor for a Static Halo}
%====================================================================
To demonstrate how the shielding factor changes as a function of halo mass and redshift, we can examine $N_{\rm H2}$ between the core and halo virial radius as a function of halo mass for an isothermal halo, with a constant \hh{} fraction, and use Equation \ref{shield} to calculate the shielding factor. The \hh{} molecule number density within the halo is given by 
\begin{equation} \label{numberh2}
	n_{\rm H2}(r) = f_{\rm H2} \frac{\rho_{0}R^{2}_{\rm vir}}{\mu m_{\rm H} r^{2}}
\end{equation}
where $\mu = 1.22$ is the mean molecular weight, $\rho_{0} = 200 \rho_{c} / 3$ is the density at the virial radius, and $\rho_{c} = 3 \rm{H}_{0}^{2} / 8 \pi \rm{G} (1+z)^3$ is the critical density. The column density is then equation \ref{numberh2} integrated from the radius of the core to the virial radius, where the core has a radius of $f_{c} R_{\rm vir}$: 
\begin{equation} \label{columnh2}
	N_{\rm H2} = f_{\rm H2} R_{\rm vir} \frac{200 \rho_{c}}{3 \mu m_{\rm H}} \left(\frac{1}{f_{c}} - 1 \right)
\end{equation}

For $\rm{H}_{0} = 70 \ \rm{km} \ \rm{s}^{-1} \ \rm{Mpc}^{-1}$, a typical \hh{} fraction of $f_{\rm H2} = 10^{-4}$, and assuming the Doppler broadening parameter, $b$, is equal to the circular velocity of the halo, Equation \ref{columnh2} and Equation \ref{shield} will give the shielding factor for a given core radius, $f_{c}$, and redshift. Figure \ref{fig:shield_mass} shows the shielding factor as a function of halo mass, for redshifts $z = 9$, $15$, and $25$ in red, blue and black, respectively. The solid lines indicate a core radius of $f_{c} = 10^{-2}$ and the dashed lines a core radius of $f_{c} = 10^{-4}$. Across the top axis and plotted as a green line, the shielding factor as a function of \hh{} column density is shown for a circular velocity of 10 km/s. At each redshift, all halo masses, and for each core radius, we see the shielding factor is significantly smaller than one, implying that \hh{} shielding for these halos is almost at a maximum. The shielding factor will then significantly decrease the dissociation rate of \hh{}, allowing for the presence of \hh{} within a LW background. Smaller mass halos can then form Pop III stars since \hh{} will survive in the halo for a longer amount of time. We also see that for column densities at just under $10^{17} \rm cm^{-2}$, the shielding factor quickly drops to 0.1 for a halo with this particular circular velocity, showing that \hh{} shielding can affect the dissociation rate of \hh{} at relatively low column densities.

\begin{figure}
	\includegraphics[width=\columnwidth]{images/shield_mass.pdf}
    \caption{Shielding factor as a function of halo mass. Solid lines indicate a core radius of $f_{c} = 10^{-2}$ and the dashed lines indicate a core radius of $f_{c} = 10^{-4}$. For these parameters, and assuming the Doppler broadening parameter equals the circular velocity, the shielding factor is almost zero, implying maximum shielding. The green line shows the shielding factor as a function of \hh{} column density for a circular velocity of 10 km/s. }
    \label{fig:shield_mass}
\end{figure}
%====================================================================
\section{Results}
%====================================================================
In this section, we will present the distribution of host halo masses of Pop III stars, the relation between the mass distribution and the LW background radiation, and the distribution of various Pop III properties.

To ensure that our box is representative of a typical piece of the universe, Figure \ref{fig:hmf} shows the simulated halo mass function along with the analytic Sheth-Tormen mass function at $z$ = 9.3. The analytic function closely matches our simulation until the mass resolution of our simulation is unable to resolve below 10$^{5}$ M$_{\odot}$ halos that contain $\approx$ 50 particles. After a sufficient amount of time after the beginning of the simulation, enough matter has collapsed to begin forming Pop III stars throughout the box. These stars either become black holes or supernova at the end of their lives, with the latter resulting in an expulsion of metals into the star's surroundings. Metal-enriched star formation begins to take place as soon as enough metals are present within a halo. In the meantime, halos form and merge with one another, resulting in a wide variety of galaxies, and reionization starts to take place. By the end of the simulation, 20\% of the box is ionized over 10\%, as can be seen in Figure \ref{fig:JLW_xe_mass}, which also shows the mass weighted LW intensity as a function of time, and temperature projections of the box at certain points in time. 

\begin{figure}
	\includegraphics[width=\columnwidth]{images/hmf.pdf}
    \caption{Halo mass function of the last output of the simulation at $z$ = 9.3. The thick, black line is the analytic Sheth-Tormen mass function. The distribution of halo masses matches well with Sheth-Tormen until about 10$^{5}$ M$_{\odot}$.}
    \label{fig:hmf}
\end{figure}

\begin{figure}
	\includegraphics[width=\columnwidth]{images/JLW_xe_mass.pdf}
    \caption{For the entire box, the average mass weighted LW background and the fraction of the volume ionized above 10\% versus time. Temperature projections are shown at the points indicated by the arrows.}
    \label{fig:JLW_xe_mass}
\end{figure}

In our simulation, 688 Pop III stars form between 27.23 $\geq z \geq$ 9.39. The corresponding star formation rate density can be seen in Figure \ref{fig:pop3_SFR_bar}. Pop III star formation peaks at $1.9 \times 10^{-4}$ at $z = 21.3$ and slowly decreases. The star formation rate density does not fall to zero at the end of the simulation, since Pop III stars were still being produced when the simulation was cut off. The SFRD plotted here is particular for our box. Due to the small size of our box, we are unable to capture all of the cosmic variance. Therefore the SFRD is likely to change depending on the size and initial conditions of the box.

\begin{figure}
	\includegraphics[width=\columnwidth]{images/pop3_SFR_bar.pdf}
    \caption{The star formation rate density of Pop III stars. The SFRD peaks at a redshift of 20, and slowly falls off. Note that metal-enriched star formation continually increases throughout time, although that is not plotted here. Due to the small box size of our simulation, this SFRD is particular for our box, and would change for larger, or smaller box sizes.}
    \label{fig:pop3_SFR_bar}
\end{figure}

%====================================================================
\subsection{Population III host halo masses}
%====================================================================
 To determine the host halos of new Pop III stars, we use the halo finding code, \rockstar{} \citep{rockstar}. The mass determined by \rockstar{} is the dark matter mass, so to determine the total mass of the halos, we multiply the masses by $\Omega_{\rm M}$/$\Omega_{\rm DM}$. The total mass is subsequently used in our analysis. We also take the center and virial radius from the \rockstar{} halo catalog. A small percentage of Pop III stars do not form in a halo identified by \rockstar{}. Assuming that Pop III stars must form in collapsed halos, we calculate the virial mass and radius of a halo centered on the Pop III star directly from the simulation data. We calculate the LW intensity impinging on each halo by summing up the contribution coming from each radiating star particle outside the virial radius of the host halo. This is then added to the constant LW background implemented in the simulation (Eq. \ref{LWbg}). In this work, we only analyze halos below the atomic cooling limit. 

In redshift bins of $\Delta z$ = 1, the mean halo mass hosting Pop III stars is determined, and plotted as black dots against the redshift bins in Figure \ref{fig:mean_mass}. The median for each bin is represented by an x, and the 15.9 and 84.1 percentiles are plotted below and above the mean, respectively. The M01 relation (Equation \ref{mthresh}) for our LW background intensity in Eq. \ref{LWbg} is shown as the black solid line. Notably, the mean halo mass falls well below the relation, at $7.7 \times 10^{5}$ M$_{\odot}$ until $z = 15$ (270 Myr), at which point, the mean halo mass rises to $3.7 \times 10^{6}$ M$_{\odot}$. The large discrepancy between our data and the mass threshold from M01 is indicative of the \hh{} shielding which is included in our simulation. \hh{} shielding allows for halos to cool at much lower masses, and therefore, Pop III stars are forming in these halos at earlier times. In the M01 analysis, \hh{} shielding is neglected in their calculations for a variety of reasons, including the Doppler broadening of LW bands and large column densities of \hh{} only beginning to form at late times. Interestingly, a full radiation hydrodynamics simulation, such as ours, does not produce identical results as M01 predicts. Throughout the literature, there appears to be a consensus that \hh{} self-shielding will help smaller mass halos collapse in the presence of a LW background radiation \citep[E.g.][]{Yoshida03, Ricotti01, Glover01, Hartwig15}. As can be seen by the red solid line in \ref{fig:mean_mass}, nearly 100\% of the halos hosting Pop III stars fall below the M01 relation, until around a redshift of 15, at which point, the mean halo mass begins to rise above the relation. The increase in the mean halo mass above the M01 relation is indicative of the increased LW intensity permeating the box due to a galaxy that becomes very active in star formation at late times. This galaxy has a halo mass of $3.6 \times 10^{8} \rm M_{\odot}$, a total stellar mass of $3.1 \times 10^{6} \rm M_{\odot}$, and has a peak Pop III SFR of $2.7 \times 10^{-4} \rm M_{\odot} yr^{-1}$ at $\approx$ 250 Myr and a peak metal enriched SFR of $5.4 \times 10^{-2}$ at $\approx$ 500 Myr. This galaxy produces a large amount of LW radiation which dominates the box, and drives up the minimum halo mass required for the formation of Pop III stars. This increase in the LW background can also be seen in Figure \ref{fig:JLW_xe_mass}, where by the end of the simulation, the LW background rises to greater than 1 $J_{21}$.

\begin{figure}
	\includegraphics[width=\columnwidth]{images/mean_mass_errorb_fix.pdf}
    \caption{For halos hosting new Pop III stars, the mean halo mass for redshift bins of $\Delta z$ = 1 is plotted along the left hand side versus time. The black scatter points indicate the mean halo mass for that redshift bin and the x indicate the median halo mass. The error bars indicate the 15.9 and 84.1 percentiles. The right hand side shows the percentage of halos that fall below the M01 mass threshold as a function of redshift. The M01 mass threshold (Eq. \ref{mthresh}), is plotted for the constant LW background in Eq. \ref{LWbg}. Almost all halos fall below the M01 relation, until $\approx$ 380 Myr, when the mean halo mass rises above the relation.}
    \label{fig:mean_mass}
\end{figure}

%====================================================================
\subsection{Lyman-Werner background for Pop III host halos}
%====================================================================

The LW background intensity versus the host halo mass for each Pop III halo in our dataset is shown in Figure \ref{fig:jlw_mass_machacek}. Each point is colored by the redshift where the Pop III star forms and the mass threshold from M01 (Eq. \ref{mthresh}) for the given LW background (Eq. \ref{LWbg}) is the solid black line. We see again that almost all halos form Pop III stars below the M01 threshold. There are also a few halos that form Pop III stars in very high J$_{\rm LW}$, below the relation. This situation arises when there are multiple Pop III stars forming at about the same time, within $\approx$ 100 pc of each other. Star formation can occur in a neighboring halo of a Pop III star whose LW radiation does not have ample time to photodissociate enough \hh{} to completely suppress star formation. The high J$_{\rm LW}$ is generally an indicator that the halos will likely have their star formation suppressed, but in cases where Pop III star formation occurs synchronically, star formation will not be suppressed, so long as they are a suitable distance away from each other. As there are only a few data points in this region, this situation is rare. It should be noted that there are duplicate halos within this plot, since halos are allowed to form multiple Pop III stars if the conditions are sufficient and each point represents an instance of Pop III formation. The grouped points in the higher end of J$_{\rm LW}$ are representative of such halos. We find that a total of 84\% of the halos forming Pop III stars lie below the M01 mass threshold over the entire simulation redshift range. We also do not see a clear relationship between the LW intensity and the host halo mass. This may be showing that the LW intensity is not as great an indicator of which halo will be allowed to form a Pop III star as previously thought. 

\begin{figure}
	\includegraphics[width=\columnwidth]{images/jlw_mass_machacek_total.pdf}
    \caption{The average LW background for host halos of new Pop III stars is plotted versus the host halo mass, colored by redshift. The Machacek et al. relation is plotted given the background LW in Eq. \ref{LWbg}. Almost all halos fall below the relation, across a range of redshifts. A few halos (or possibly a single halo) allow Pop III stars to form at a low halo mass in a very high LW background. There are a few halos that have Pop III stars forming at later times and at high halo masses.}
    \label{fig:jlw_mass_machacek}
\end{figure}

%====================================================================
\subsection{Multiple stellar systems}
%====================================================================

We now inspect the number of Pop III stars and the total mass of Pop III stars per halo. Again, we are only tracking massive stars, not any stars that may form out of fragmented gas with masses < $1 M_{\odot}$. The left panel of Figure \ref{fig:num_p3} shows the number of Pop III stars per halo for a given halo mass, just after star formation. The histogram of the number of Pop III stars for all masses is projected on the right hand side. We find that a median number of four Pop III stars form per halo, with a maximum of 16 Pop III stars forming per halo. Since we did not restrict the number of Pop III stars that can form in a halo, we find that the conditions are often sufficient for multiple Pop III stars to form in a single halo. Out of the halos forming Pop III stars, only 16\% of them form single Pop III stars, whereas 54\% form between two and five Pop III stars. It should be noted that these halos are not necessarily forming all of the Pop III stars at once. For example, a halo can form stars again if it did not host a supernova. The number of Pop III stars that form within a halo is indicative of the star formation history of that halo. Figure \ref{fig:totp3mass_halomass_sidehist} shows the total mass of Pop III stars per halo for a given halo mass, where the histogram of the total Pop III mass for all halos is projected on the right hand side. Lines of constant star formation efficiency are overplotted. We find that most of our Pop III stars are forming between 10$^{-4}$ < f$_\star$ < 10$^{-3}$. The mean total mass of Pop III stars for all halos is 195 M$_{\odot}$. Note that the mass of the Pop III stars depends on the chosen IMF. 

At the end of the simulation, the right panel of Figure \ref{fig:num_p3} shows the distribution of the number of Pop III stars per halo for a given halo mass, for all halos in the simulation that host either a living or a remnant Pop III star. The red line shows the median number of Pop III stars for each halo mass bin. At low masses, the number of Pop III stars per halo sits around 2 Pop III stars, and quickly rises once the halo mass reaches $10^{7} M_{\odot}$. \citet{Xu13} found that at $z = 15$, halos with masses at about $10^{7} M_{\odot}$ will contain between 10 and 100 living Pop III stars, with even more remnants. While we do not see quite as many Pop III stars, we agree that high mass halos can contain a large amount of Pop III stars. We can also look at the spread in creation times of these Pop III stars per halo, shown in Figure \ref{fig:p3spread_mass}. The halos that have a zero spread in their Pop III creation time, meaning the Pop III stars formed at the same time, lie at $10^{-2}$ Myr. The red line indicates the median spread in creation times for each halo mass bin. There are two groups within this plot. The first group contains halos that form their Pop III stars in a spread of < 10 Myr, with a median spread of about $10^{-1}$ Myr. The second group contains halos that form their Pop III stars in a spread of > 10 Myr, with a median spread of about 100 Myr. The second group represents typically larger mass halos that have acquired older Pop III stars by the end of the simulation through their long merger history. These halos have acquired these Pop III stars from outside halos, accounting for the large spread in Pop III creation time as well as the increased number of Pop III stars, as seen in the right panel of Figure \ref{fig:num_p3}. 

\begin{figure*}
  \centering
  \includegraphics[width=0.48\textwidth]{images/totnump3_halomass_sidehist.pdf}
  \hfill
  \includegraphics[width=0.48\textwidth]{images/final_redshift_Np3_mass.pdf}
  \caption{The number of Pop III stars per halo versus halo mass at the instance of Pop III star formation (left) and the final redshift (right). At the instance of formation, Pop III stars will form in a halo which contains a median number of four Pop III stars, with some containing as many as 16 Pop III stars. On the right, the red line indicates the median number of Pop III stars in each halo mass bin. By the end of the simulation, high mass halos contain a large number of Pop III stars due to their long merger history with smaller halos hosting Pop III stars.}
  \label{fig:num_p3}
\end{figure*}


\begin{figure}
	\includegraphics[width=\columnwidth]{images/totp3mass_halomass_sidehist.pdf}
    \caption{Total mass of Pop III stars in halos hosting new Pop III stars versus halo mass. Lines of constant star formation efficiencies are overplotted. Most halos form Pop III stars at efficiencies between 10$^{-3}$ and 10$^{-4}$.}
    \label{fig:totp3mass_halomass_sidehist}
\end{figure}

\begin{figure}
	\includegraphics[width=\columnwidth]{images/p3spread_mass.pdf}
    \caption{The spread in Pop III creation times per halo versus halo mass at the end of the simulation. The red line indicates the median spread in creation times for each halo mass bin. There are two groups that appear in this plot, halos that have a spread of less than 10 Myr, representing halos that form Pop III stars quickly, and halos that have a spread of greater than 10 Myr, representing larger mass halos that acquire old Pop III stars through mergers.}
    \label{fig:p3spread_mass}
\end{figure}

%====================================================================
\section{Discussion}
%====================================================================

%====================================================================
\subsection{Variations in the halo masses at collapse}
%====================================================================
Previous studies have focused on the minimum halo mass of Pop III hosts as a function of LW intensity. However, they have also found that Pop III stars often form in halos up to an order of magnitude greater than the mininum. We also have found that Pop III star formation occurs in a similar range of host halo masses throughout our simulation. There are three main causes of this significant variation in mass. First, a fraction of metal-free stars directly form a black hole without metal ejecta \citep{Heger03}, leaving the halo chemically pristine.  Second-generation stars forming later in that halo could still be metal-free if no external enrichment occurs. Thus they would form in halos substantially larger than the minimum, as it takes tens of millions of years for the halo to recover its gas after radiative feedback \citep{Muratov13, Jeon14_Recovery}.  Second, dynamical heating from mergers and accretion provide a heating source inside the growing halo, preventing the gas from efficiently cooling via \hh{}. This turbulent stirring can delay star formation that could cool through \hh{} in an ideal situtation but otherwise forms in halos more massive than the minimum \citep{Yoshida03, Wise19}. Finally, temporal fluctuations in the local LW radiation field can greatly affect the amount of \hh{} within a halo and thus, how efficiently the halo can cool. While this may only be a small effect in some halos where the local LW radiation flucutations are relatively small and star formation is distant, Pop III star formation may be significantly delayed or completely prevented in halos that are close to active star formation sites.  In summary, the timing of Pop III star formation depends on both local -- halo histories of star formation and growth -- and environmental properties.  The combination of these three processes determines which halos will be able to form Pop III stars, and thus results in a wide range of host halo masses. 


%====================================================================
\subsection{Comparison to previous work}
%====================================================================

There has been a large amount of work done throughout the literature investigating the relationship between the LW background intensity and halo masses. \citet{Tegmark97} numerically integrated a chemical network to calculate the minimum mass a halo must have in order to cool, by considering \hh{} formation and cooling rates. They conclude that cooling by \hh{} is efficient and leads to the first formation of structures. They found a minimum halo mass needed to collapse which depends on the virial temperature of the halo and the virial redshift. At $z = 20$, this minimum halo mass is $4 \times 10^{6} M_{\odot}$. \citet{Trenti09} uses a similar argument as \citet{Tegmark97}, and found that metal-free halos can exist until z $\approx$ 6 using cosmological simulations and an analytical model for metal enrichment. They also provide some insight into whether or not Pop III supernova rates are high enough within these metal-free halos to allow for the LSST to see these regions of the universe. They used the minimum halo mass capable of cooling via \hh{} from \citet{Trenti09_SFR} as one part of their halo mass model (see the blue solid line in Figure \ref{fig:compare_JLW_mass}, given our LW background). Neither \citet{Trenti09} nor \citet{Tegmark97} included the effects of \hh{} self-shielding, which can explain the discrepancy between our data and their mass threshold relationship (Figure \ref{fig:compare_JLW_mass}). For a LW background of $J_{LW} / J_{21} = 0.2$, we see halos hosting Pop III stars at $\approx 4 \times 10^{5}$ M$_{\odot}$, while the relation from \citet{Trenti09_SFR} would expect to see minimum halo masses capable of cooling, and thus capable of hosting Pop III stars, at $10^7$ M$_{\odot}$, almost two orders of magnitude larger. Since we include \hh{} self-shielding, Pop III stars are allowed to form in smaller mass halos for a given LW background. \citet{Mebane18} used a semi-analytic model of star formation, including feedback properties such as a LW background, photoionization due to Pop III stars, supernovae of Pop III stars and metal-enriched stars, and chemical enrichment, to determine for how long Pop III stars will survive for and found that Pop III stars can continue to form until z $\approx$ 6. They found a minimum halo mass for hosting Pop III star formation, but with the caveat that they did not include \hh{} self-shielding. They found that when Pop III stars contribute to the LW background, the minimum halo mass for Pop III star formation is $4 \times 10^{6} M_{\odot}$ at $z = 20$, similar to results from \citet{Tegmark97}. The results from \citet{Trenti09} and \citet{Mebane18} have higher halo masses in comparison with M01, although this more closely matches results from simulations \citep[see][]{Wise07_UVB, OShea08}. It should be noted that because our box is fairly small, we cannot capture the cosmic variance of rare halos and galaxies. For example, at late times, a single galaxy dominates the LW radiation field, and drives up the host halo masses. However, we would expect this to happen at different times in other cosmological volumes. Therefore, we cannot directly compare the time dependence, however, a comparison as a function of LW intensity is still valid. 

\citet{Yoshida03} used cosmological simulations to study the formation of primordial star-forming clouds. They followed the growth of structure to find where gas cools and condenses which would form the first stars, and how the effects of LW radiation may affect these gas clouds. Importantly, they included \hh{} self-shielding within halos. In a series of simulations, they found a minimum halo mass for those halos hosting gas clouds which may result in active star formation (see their Figure 12). They found that in the presence of a LW background of J$_{21}$ = 0.01, the minimum halo mass is $7 \times 10^{5} M_{\odot}$. When \hh{} self-shielding is taken into account along with a LW background, their minimum halo mass decreases to $4.5 \times 10^{5} M_{\odot}$. This value lies close to the case where there is no LW background applied, where the minimum halo mass is $3.5 \times 10^{5} M_{\odot}$. They find that \hh{} self-shielding does appear to be an efficient mechanism for primordial gas cooling. In comparison with our work, we find similar minimum halo masses for each redshift, although we do see a wider range of halo masses across redshifts, ranging from $2.8 \times 10^{5}$ M$_{\odot}$ to $1.4 \times 10^{7}$ M$_{\odot}$. While our halos never experience a LW intensity as low as 0.01, we find that their minimum mass for this LW intensity lies in the same mass range as our halos (see blue open circle in Figure \ref{fig:compare_JLW_mass}). \citet{Wise07_UVB} used cosmological simulations to investigate \hh{} cooling in a LW background. They found that \hh{} cooling is dominant even when there is a large LW background present. They did not include self-shielding and subsequently found halo masses at their collapse that lie well above the M01 relation (see red open triangles in Figure \ref{fig:compare_JLW_mass}). Their J$_{LW}$ = 0 control is plotted in Figure \ref{fig:compare_JLW_mass} at J$_{LW}$/J$_{21}$ = 10$^{-4}$. \citet{OShea08} used cosmological simulations to investivate Pop III star formation in various LW backgrounds. They found that due to an increased LW background, there is a delay in star formation, and thus there is an increase in the halo masses at collapse. They also ignored \hh{} self-shielding in their calculations which can again account for the increased halo masses they found compared to this work, as can be seen in Figure \ref{fig:compare_JLW_mass} (open green squares). Their halo masses are similar to the results of \citet{Wise07_UVB}. Their control of J$_{21}$ = 0 control is plotted at J$_{LW}$/J$_{21}$ = 10$^{-4}$.

\begin{figure}
	\includegraphics[width=\columnwidth]{images/compare_JLW_mass.pdf}
    \caption{The average LW intensity normalized by J21 versus the halo mass for this work and a variety of other works. The black solid line is from M01, blue solid line from \citet{Trenti09_SFR}, blue open circle from \citet{Yoshida03}, green open squares from \citet{OShea08}, and red open triangles from \citet{Wise07_UVB}. J$_{LW}$ = 0 for the various sources is plotted here at J$_{LW}$/J$_{21}$ = 10$^{-4}$. The solid bands of halo masses correspond to the characteristic masses of halos to contain a baryon fraction that is half of the cosmic mean for various redshifts and for the shown streaming velocities from \citet{Naoz13}. For each band, the maximum halo mass occurs at $z = 15$. For streaming velocities of $3.4 \sigma_{\rm vbc}$, $1.7 \sigma_{\rm vbc}$, and $1 \sigma_{\rm vbc}$, the minimum halo mass occurs at $z = 17, 22, \rm{and} \ 27$, respectively.}
    \label{fig:compare_JLW_mass}
\end{figure}

%====================================================================
\subsection{Caveats}
%====================================================================

There are a few minor shortcomings to this work that should be noted. \citet{Schauer17} investigated the escape fraction of LW photons in the near and far-field and found that the LW escape fraction of atomic cooling halos can vary significantly depending on the ionisation front within the halo. In the near-field with an ionisation front breaking out of a halo, they find that the LW escape fraction is greater than 95\%. But when the ionisation front is contained within the halo, they found that the LW escape fraction can range from 3\% to 88\%. In general, they find that the LW escape fractions in the far-field are higher than 75\%. Since we are not accounting for the reduced escape fraction within our halos, we are overestimatng the LW radiation coming from point sources in our simulation. We could account for this by decreasing the intensity of the stars, since the overall LW radiation would be reduced, although this should not significantly affect our results as we find little dependence on LW intensity. 

We also do not include streaming velocities between baryons and dark matter within our simulation \citep{Tselia11, Greif11_Delay, Naoz12, OLeary12}. Streaming velocities can suppress star formation at very high ($z \ga 20$) redshifts only, because the streaming velocities decrease with time. Pop III star formation is delayed by an average of $\delta z = 4$ by increasing the halo mass needed to overcome the bulk velocity by about a factor of three \citep{Greif11_Delay}. The streaming velocities will generally result in lower gas densities within halos as well as an offset of the peak density from the center of the halo \citep{OLeary12}. Often in the literature, streaming velocities are only studied for a few halos. But \citet{Naoz13} estimated the minimum mass needed to retain the bulk of the baryons within the dark matter halo using a series of cosmological simulations, the results of which can be seen in Figure \ref{fig:compare_JLW_mass} (see shaded bands). These halos only reach up to about $2 - 3 \times 10^5 M_\odot$ when the streaming velocity is about $1\sigma_{\rm vbc}$, which is smaller than the star forming halos in our simulation. \citet{Naoz13} mention that at the largest streaming velocity, their simulations do not match expected values due to a small sample size of high mass halos. If we had included streaming velocities, the gas would have already fallen into the dark matter potential well when the halo exceeds this characteristic mass. Because of this, streaming velocities would likely not affect our results. Finally, there is the obvious caveat of the unknown Pop III IMF. Like with any work done assuming a Pop III IMF, uncertainties about metal enrichment, multiplicity, total stellar mass, ionization rates, supernova rates, and the amount of black holes produced are introduced into our results.

%====================================================================
\section{Conclusions}
%====================================================================
In this work, we presented the analysis of the host halo mass distribution of Pop III stars and how it relates to the LW background radiation. From our results, we conclude the following: 

\begin{itemize}
	\item We find that Pop III stars are forming in halos with a mean mass of $7.7 \times 10^{5}$ M$_{\odot}$, which falls well below the M01 relation, until $\approx$ 380 Myr, at which point the mean halo mass rises above the M01 relation to a mean value of $3.7 \times 10^{6}$ M$_{\odot}$ due to the domination of LW radiation produced by metal-enriched stars in the most massive galaxy in the simulation box.
	\item Including \hh{} self-shielding results in a factor of up to 1.1 lower mass halos compared to M01, forming Pop III stars.
	\item There does not appear to be a strict correlation between LW intensity within a halo and the halo mass. LW intensity within a halo may not be as indicative of which halos may form Pop III stars as previously thought.
	\item Halos are likely to form multiple Pop III stars. At the instance of Pop III star formation, a halo will have formed a median number of four Pop III stars, up to a maximum of 16.
	\item By the end of the simulation, halos with masses > $10^{7} M_{\odot}$ acquire multiple Pop III stars due to mergers with smaller mass halos. This results in these high mass halos containing multiple Pop III stars, both young and old. 
\end{itemize}

Our results provide another piece of the Pop III puzzle. With the inclusion of \hh{} self-shielding, we find Pop III stars forming in smaller mass halos than previously predicted. These smaller mass halos provide earlier and additional sites for Pop III star formation, and therefore, more metal-enrichment within halos. We also find that Pop III stars do not necessarily have to form in isolation, in fact, they are more likely to form in a halo that has already formed multiple other Pop III stars. Since these stars are generally massive and are likely to produce stellar mass black holes, their remnants may be gravitational wave candidates detected through LIGO \citep{Hartwig16}. The exact detection signature of these gravitational waves is a topic which needs to be studied further. While we do not see any relationship between the LW intensity and Pop III host halo mass,  in order to determine exactly how the LW intensity influences the host halo mass, a more systematic approach needs to be taken. 

The assumptions often made about the formation of Pop III stars, like how they will only form in isolation and how \hh{} self-shielding will not affect formation, appear to break down when self-shielding is taken into account and restrictions about their formation are lifted. Further work needs to be done to determine why some halos form Pop III stars and others do not, to provide insight about the formation process of these stars and thus how they will affect their surrounding galactic environment. The effect of the LW intensity on host halo mass should be systematically investigated to determine whether or not this background radiation really has a profound effect on Pop III formation or not. These primordial objects need to be studied further to assist multi-messenger astronomers in finding these ancient systems. 

%====================================================================
\section*{Acknowledgements}
%====================================================================

JHW is supported by National Science Foundation grants AST-1614333 and OAC-1835213, NASA grant NNX17AG23G, and Hubble theory grant
HST-AR-14326.  We thank the support staff at Georgia Tech's PACE,
where we ran this simulation.  The freely available plotting library
{\sc matplotlib} \citep{matplotlib} was used to construct numerous
plots within this paper. Computations and analysis described in this
work were performed using the publicly-available \enzo{} and \yt{}
codes, which is the product of a collaborative effort of many
independent scientists from numerous institutions around the world.

%%%%%%%%%%%%%%%%%%%%%%%%%%%%%%%%%%%%%%%%%%%%%%%%%%

%%%%%%%%%%%%%%%%%%%% REFERENCES %%%%%%%%%%%%%%%%%%

% The best way to enter references is to use BibTeX:

\bibliographystyle{mnras}
\bibliography{jwise} % if your bibtex file is called example.bib


% Alternatively you could enter them by hand, like this:
% This method is tedious and prone to error if you have lots of references
%\begin{thebibliography}{99}
%\end{thebibliography}

%%%%%%%%%%%%%%%%%%%%%%%%%%%%%%%%%%%%%%%%%%%%%%%%%%

%%%%%%%%%%%%%%%%% APPENDICES %%%%%%%%%%%%%%%%%%%%%

\appendix

%%%%%%%%%%%%%%%%%%%%%%%%%%%%%%%%%%%%%%%%%%%%%%%%%%


% Don't change these lines
\bsp	% typesetting comment
\label{lastpage}
\end{document}

% End of mnras_template.tex