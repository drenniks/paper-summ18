% paper.tex
%
% LaTeX template for creating an MNRAS paper
%
% v3.0 released 14 May 2015
% (version numbers match those of mnras.cls)
%
% Copyright (C) Royal Astronomical Society 2015
% Authors:
% Keith T. Smith (Royal Astronomical Society)

% Change log
%
% v3.0 May 2015
%    Renamed to match the new package name
%    Version number matches mnras.cls
%    A few minor tweaks to wording
% v1.0 September 2013
%    Beta testing only - never publicly released
%    First version: a simple (ish) template for creating an MNRAS paper

%%%%%%%%%%%%%%%%%%%%%%%%%%%%%%%%%%%%%%%%%%%%%%%%%%
% Basic setup. Most papers should leave these options alone.
\documentclass[a4paper,fleqn,usenatbib]{mnras}

% MNRAS is set in Times font. If you don't have this installed (most LaTeX
% installations will be fine) or prefer the old Computer Modern fonts, comment
% out the following line
\usepackage{newtxtext,newtxmath}
% Depending on your LaTeX fonts installation, you might get better results with one of these:
%\usepackage{mathptmx}
%\usepackage{txfonts}

% Use vector fonts, so it zooms properly in on-screen viewing software
% Don't change these lines unless you know what you are doing
\usepackage[T1]{fontenc}
\usepackage{ae,aecompl}


%%%%% AUTHORS - PLACE YOUR OWN PACKAGES HERE %%%%%

% Only include extra packages if you really need them. Common packages are:
\usepackage{graphicx}	% Including figure files
\usepackage{amsmath}	% Advanced maths commands
\usepackage{amssymb}	% Extra maths symbols

%%%%%%%%%%%%%%%%%%%%%%%%%%%%%%%%%%%%%%%%%%%%%%%%%%

%%%%% AUTHORS - PLACE YOUR OWN COMMANDS HERE %%%%%

\newcommand{\fesc}{\ifmmode{f_{\rm esc}}\else{$f_{\rm esc}$}\fi}
\newcommand{\fescs}{\ifmmode{f_{\rm esc}^\star}\else{$f_{\rm esc}^\star$}\fi}
\newcommand{\kms}{\ifmmode{{\;\rm km~s^{-1}}}\else{km~s$^{-1}$}\fi}
\newcommand{\fgas}{\ifmmode{{f_{\rm gas}}}\else{$f_{\rm gas}$}\fi}
\newcommand{\cubecm}{\ifmmode{{\rm cm^{-3}}}\else{cm$^{-3}$}\fi}
\newcommand{\ztwo}{\ifmmode{{\rm [Z_2/H]}}\else{[Z$_2$/H]}\fi}
\newcommand{\zthree}{\ifmmode{{\rm [Z_3/H]}}\else{[Z$_3$/H]}\fi}
\newcommand{\lsim}{\lower0.3em\hbox{$\,\buildrel <\over\sim\,$}}
\newcommand{\gsim}{\lower0.3em\hbox{$\,\buildrel >\over\sim\,$}}
\newcommand{\li}{\noindent$\bullet$\quad}
\newcommand{\lli}{$\bullet$\quad}
\newcommand{\lcdm}{$\Lambda$CDM}
\newcommand{\flux}{erg s$^{-1}$ cm$^{-2}$ Hz$^{-1}$}
\newcommand{\emis}{erg s$^{-1}$ cm$^{-2}$ Hz$^{-1}$ sr$^{-1}$}
\newcommand{\sfr}{\ifmmode{\textrm{M}_\odot \,\textrm{yr}^{-1} \,\textrm{Mpc}^{-3}}\else{M$_\odot$ yr$^{-1}$ Mpc$^{-3}$}\fi}
\newcommand{\hsfr}{\ifmmode{\textrm{M}_\odot\, \textrm{yr}^{-1}}\else{M$_\odot$ yr$^{-1}$}\fi}
\newcommand{\ssfr}{Gyr$^{-1}$}
\newcommand{\arp}{$a_{\rm rp}$}
\newcommand{\agrav}{$a_{\rm grav}$}
\newcommand{\eavg}{\ifmmode{\langle E_\gamma \rangle}\else{$\langle E_\gamma \rangle$}\fi}
\newcommand{\hst}{{\it HST}}
\newcommand{\enzo}{{\sc enzo}}
\newcommand{\yt}{{\sc yt}}
\newcommand{\moray}{{\sc enzo+moray}}
\newcommand{\Ms}{\ifmmode{M_\odot}\else{$M_\odot$}\fi}
\newcommand{\vrms}{\ifmmode{v_{\rm rms}}\else{$v_{\rm rms}$}\fi}
\newcommand{\mmin}{M$_{min}$}
\newcommand{\hh}{H$_2$}
\newcommand{\Ol}{$\Omega_\Lambda$}
\newcommand{\Om}{$\Omega_M$}
\newcommand{\Ob}{$\Omega_b$}
\newcommand{\theat}{$t_{\rm{heat}}$}
\newcommand{\tcool}{$t_{\rm{cool}}$}
\newcommand{\rcool}{$r_{\rm{cool}}$}
\newcommand{\tcross}{$t_{\rm{cross}}$}
\newcommand{\tdyn}{$t_{\rm{dyn}}$}
\newcommand{\tkh}{$t_{\rm{KH}}$}
\newcommand{\tH}{$t_{\rm{H}}$}
\newcommand{\tvir}{\ifmmode{T_{\rm{vir}}}\else{$T_{\rm{vir}}$}\fi}
\newcommand{\mvir}{\ifmmode{M_{\rm{vir}}}\else{$M_{\rm{vir}}$}\fi}
\newcommand{\rvir}{\ifmmode{r_{\rm{vir}}}\else{$r_{\rm{vir}}$}\fi}
\newcommand{\rr}{$r_{200}$}
\newcommand{\lya}{Ly$\alpha$}
\newcommand{\jj}{\ifmmode{J_{21}}\else{$J_{21}$}\fi}
\newcommand{\flw}{\ifmmode{F_{LW}}\else{$F_{LW}$}\fi}
\newcommand{\kph}{\ifmmode{k_{\rm ph}}\else{$k_{\rm ph}$}\fi}
\newcommand{\tv}{$\langle T \rangle_{\rm v}$}
\newcommand{\tm}{$\langle T \rangle_{\rm m}$}
\newcommand{\msun}{{\rm\,M_\odot}} 
\newcommand{\lsun}{{\rm\,L_\odot}}
\newcommand{\zsun}{\ifmmode{\rm\,Z_\odot}\else{$\rm\,Z_\odot$}\fi}
\newcommand{\etal}{et al.\ }
\newcommand\tento[1]{$10^{#1}$}
\newcommand\step[1]{\textit{Step #1}.--}
\newcommand\halo[1]{\textit{Halo #1}.--}
\newcommand{\hi}{H {\sc i}}
\newcommand{\hii}{H {\sc ii}}
\newcommand{\hei}{He {\sc i}}
\newcommand{\heii}{He {\sc ii}}
\newcommand{\heiii}{He {\sc iii}}
\newcommand{\nhi}{\ifmmode{N_{\rm HI}}\else{$N_{\rm HI}$}\fi}
\newcommand\unit[1]{\; \textrm{#1}}
\newcommand{\pem}{\unit{s}^{-1} \unit{cMpc}^{-3}}
\newcommand{\music}{{\sc Music}}
\newcommand{\rockstar}{{\sc Rockstar}}

\newcommand{\refs}{(\textbf{Add references})}
\newcommand{\jhw}[1]{{\color{red} \bf (JHW: #1)}}


%%%%%%%%%%%%%%%%%%%%%%%%%%%%%%%%%%%%%%%%%%%%%%%%%%
%% Additions from AASTeX
%%%%%%%%%%%%%%%%%%%%%%%%%%%%%%%%%%%%%%%%%%%%%%%%%%

%\newcommand\ion[2]{#1$\;${\small\rmfamily\@Roman{#2}}\relax}
\def\eps@scaling{1.0}% 
\newcommand\epsscale[1]{\gdef\eps@scaling{#1}}% 
\newcommand\plotone[1]{% 
 \centering 
 \leavevmode 
 \includegraphics[width={\eps@scaling\columnwidth}]{#1}% 
}% 
\newcommand\plottwo[2]{% 
 \centering 
 %\leavevmode 
 %\columnwidth=.5\columnwidth 
 \includegraphics[width={\eps@scaling\columnwidth}]{#1}% 
 \hfil 
 \includegraphics[width={\eps@scaling\columnwidth}]{#2}% 
}% 

% \newcommand{\nat}{Nature}
% \newcommand{\aj}{AJ}
% \newcommand{\apj}{ApJ}
% \newcommand{\apjl}{ApJL}
% \newcommand{\apjs}{ApJS}
% \newcommand{\mnras}{MNRAS}
% \newcommand{\aap}{A\&A}
% \newcommand{\pasj}{PASJ}
% \newcommand{\pasp}{PASP}
% \newcommand{\araa}{ARA\&A}
% \newcommand{\bain}{Bull.~Astron.~Inst.~Netherlands}
% \newcommand{\memsai}{Mem.~Soc.~Astron.~Italiana}
% \newcommand{\physrep}{Physics Reports}
% \newcommand{\prd}{Phys.~Rev.~D}
% \newcommand{\jcap}{JCAP}

\hyphenation{sSFR sSFRs}


% Please keep new commands to a minimum, and use \newcommand not \def to avoid
% overwriting existing commands. Example:
%\newcommand{\pcm}{\,cm$^{-2}$}	% per cm-squared

%%%%%%%%%%%%%%%%%%%%%%%%%%%%%%%%%%%%%%%%%%%%%%%%%%

%%%%%%%%%%%%%%%%%%% TITLE PAGE %%%%%%%%%%%%%%%%%%%

% Title of the paper, and the short title which is used in the headers.
% Keep the title short and informative.
\title[Short title, max. 45 characters]{MNRAS \LaTeXe\ template -- title goes here}

% The list of authors, and the short list which is used in the headers.
% If you need two or more lines of authors, add an extra line using \newauthor
\author[Danielle Skinner et al.]{
Danielle Skinner,$^{1}$\thanks{E-mail: drenniks@gatech.edu}
John H. Wise,$^{1}$
\\
% List of institutions
$^{1}$Center for Relativistic Astrophysics, Georgia Institute of Technology, Atlanta, GA 30332, USA\\
}

% These dates will be filled out by the publisher
%\date{Accepted XXX. Received YYY; in original form ZZZ}

% Enter the current year, for the copyright statements etc.
\pubyear{2018}

% Don't change these lines
\begin{document}
\label{firstpage}
\pagerange{\pageref{firstpage}--\pageref{lastpage}}
\maketitle

% Abstract of the paper
\begin{abstract}
This is a simple template for authors to write new MNRAS papers.
The abstract should briefly describe the aims, methods, and main results of the paper.
It should be a single paragraph not more than 250 words (200 words for Letters).
No references should appear in the abstract.
\end{abstract}

% Select between one and six entries from the list of approved keywords.
% Don't make up new ones.
\begin{keywords}
keyword1 -- keyword2 -- keyword3
\end{keywords}

%%%%%%%%%%%%%%%%%%%%%%%%%%%%%%%%%%%%%%%%%%%%%%%%%%

%%%%%%%%%%%%%%%%% BODY OF PAPER %%%%%%%%%%%%%%%%%%

\section{Introduction}

\li Here are some relevant papers that studied the minimum halo mass
with simulations or ones that developed an analytical model of the
halo mass.  We will probably cite most of these in the introduction.

\lli Semi-analytic papers: \citep{Tegmark97, Trenti09, Visbal18,
  Mebane18, Griffen18}

\lli Simulation papers: \citep{Machacek01, Yoshida03, Wise07_UVB,
  OShea08, Muratov13}

\lli {\it Note:} I looked at which papers cited the \citet{Machacek01}
paper to look for recent papers.  I didn't find any in the past five
years that used simulations to study the minimum halo mass.



\section{Methods}

\subsection{Simulation setup}

We used Enzo \citep{Enzo}.

\subsection{Star formation and feedback}

\section{Results}

\subsection{Population III host halo masses}

\subsection{Multiple stellar systems}

\subsection{Population III stars in atomic cooling halos}

\section{Discussion}

\subsection{Variations in the halo masses at collapse}

\li Halos can form metal-free stars again if their progenitors had
only formed stars that did not go supernova

\li Dynamical heating \citep{Yoshida03}

\li Temporal fluctuations in the local Lyman-Werner radiation

\subsection{Comparison to previous work}

\li Semi-analytic papers: \citep{Tegmark97, Trenti09, Visbal18,
  Mebane18, Griffen18}

\li Simulation papers: \citep{Machacek01, Yoshida03, Wise07_UVB,
  OShea08, Muratov13}

\li We don't have to compare with all of the semi-analytic papers, but
it will be good to compare with all of the simulation papers.

\subsection{Caveats}

\li Optically-thin Lyman-Werner radiation field from point sources \citep{Schauer17}

\li No streaming velocities that suppresses star formation at very
high ($z \ga 20$) redshifts \citep{Tselia11, Greif11_Delay, Naoz12, OLeary12}

\li Uncertainties from the assumed Pop III initial mass function 

\section{Conclusions}


\section*{Acknowledgements}


%%%%%%%%%%%%%%%%%%%%%%%%%%%%%%%%%%%%%%%%%%%%%%%%%%

%%%%%%%%%%%%%%%%%%%% REFERENCES %%%%%%%%%%%%%%%%%%

% The best way to enter references is to use BibTeX:

\bibliographystyle{mnras}
\bibliography{jwise} % if your bibtex file is called example.bib


% Alternatively you could enter them by hand, like this:
% This method is tedious and prone to error if you have lots of references
%\begin{thebibliography}{99}
%\end{thebibliography}

%%%%%%%%%%%%%%%%%%%%%%%%%%%%%%%%%%%%%%%%%%%%%%%%%%

%%%%%%%%%%%%%%%%% APPENDICES %%%%%%%%%%%%%%%%%%%%%

\appendix

%%%%%%%%%%%%%%%%%%%%%%%%%%%%%%%%%%%%%%%%%%%%%%%%%%


% Don't change these lines
\bsp	% typesetting comment
\label{lastpage}
\end{document}

% End of mnras_template.tex