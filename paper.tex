% paper.tex
%
% LaTeX template for creating an MNRAS paper
%
% v3.0 released 14 May 2015
% (version numbers match those of mnras.cls)
%
% Copyright (C) Royal Astronomical Society 2015
% Authors:
% Keith T. Smith (Royal Astronomical Society)

% Change log
%
% v3.0 May 2015
%    Renamed to match the new package name
%    Version number matches mnras.cls
%    A few minor tweaks to wording
% v1.0 September 2013
%    Beta testing only - never publicly released
%    First version: a simple (ish) template for creating an MNRAS paper

%%%%%%%%%%%%%%%%%%%%%%%%%%%%%%%%%%%%%%%%%%%%%%%%%%
% Basic setup. Most papers should leave these options alone.
\documentclass[a4paper,fleqn,usenatbib]{mnras}

% MNRAS is set in Times font. If you don't have this installed (most LaTeX
% installations will be fine) or prefer the old Computer Modern fonts, comment
% out the following line
\usepackage{newtxtext,newtxmath}
% Depending on your LaTeX fonts installation, you might get better results with 
% one of these:
%\usepackage{mathptmx}
%\usepackage{txfonts}

% Use vector fonts, so it zooms properly in on-screen viewing software
% Don't change these lines unless you know what you are doing
\usepackage[T1]{fontenc}
\usepackage{ae,aecompl}


%%%%% AUTHORS - PLACE YOUR OWN PACKAGES HERE %%%%%

% Only include extra packages if you really need them. Common packages are:
\usepackage{graphicx}	% Including figure files
\usepackage{amsmath}	% Advanced maths commands
\usepackage{amssymb}	% Extra maths symbols
\usepackage{mathtools}

%%%%%%%%%%%%%%%%%%%%%%%%%%%%%%%%%%%%%%%%%%%%%%%%%%

%%%%% AUTHORS - PLACE YOUR OWN COMMANDS HERE %%%%%

\newcommand{\fesc}{\ifmmode{f_{\rm esc}}\else{$f_{\rm esc}$}\fi}
\newcommand{\fescs}{\ifmmode{f_{\rm esc}^\star}\else{$f_{\rm esc}^\star$}\fi}
\newcommand{\kms}{\ifmmode{{\;\rm km~s^{-1}}}\else{km~s$^{-1}$}\fi}
\newcommand{\fgas}{\ifmmode{{f_{\rm gas}}}\else{$f_{\rm gas}$}\fi}
\newcommand{\cubecm}{\ifmmode{{\rm cm^{-3}}}\else{cm$^{-3}$}\fi}
\newcommand{\ztwo}{\ifmmode{{\rm [Z_2/H]}}\else{[Z$_2$/H]}\fi}
\newcommand{\zthree}{\ifmmode{{\rm [Z_3/H]}}\else{[Z$_3$/H]}\fi}
\newcommand{\lsim}{\lower0.3em\hbox{$\,\buildrel <\over\sim\,$}}
\newcommand{\gsim}{\lower0.3em\hbox{$\,\buildrel >\over\sim\,$}}
\newcommand{\li}{\noindent$\bullet$\quad}
\newcommand{\lli}{$\bullet$\quad}
\newcommand{\lcdm}{$\Lambda$CDM}
\newcommand{\flux}{erg s$^{-1}$ cm$^{-2}$ Hz$^{-1}$}
\newcommand{\emis}{erg s$^{-1}$ cm$^{-2}$ Hz$^{-1}$ sr$^{-1}$}
\newcommand{\sfr}{\ifmmode{\textrm{M}_\odot \,\textrm{yr}^{-1} \,\textrm{Mpc}^{-3}}\else{M$_\odot$ yr$^{-1}$ Mpc$^{-3}$}\fi}
\newcommand{\hsfr}{\ifmmode{\textrm{M}_\odot\, \textrm{yr}^{-1}}\else{M$_\odot$ yr$^{-1}$}\fi}
\newcommand{\ssfr}{Gyr$^{-1}$}
\newcommand{\arp}{$a_{\rm rp}$}
\newcommand{\agrav}{$a_{\rm grav}$}
\newcommand{\eavg}{\ifmmode{\langle E_\gamma \rangle}\else{$\langle E_\gamma \rangle$}\fi}
\newcommand{\hst}{{\it HST}}
\newcommand{\enzo}{{\sc enzo}}
\newcommand{\yt}{{\sc yt}}
\newcommand{\moray}{{\sc enzo+moray}}
\newcommand{\Ms}{\ifmmode{M_\odot}\else{$M_\odot$}\fi}
\newcommand{\vrms}{\ifmmode{v_{\rm rms}}\else{$v_{\rm rms}$}\fi}
\newcommand{\mmin}{M$_{min}$}
\newcommand{\hh}{H$_2$}
\newcommand{\Ol}{$\Omega_\Lambda$}
\newcommand{\Om}{$\Omega_M$}
\newcommand{\Ob}{$\Omega_b$}
\newcommand{\theat}{$t_{\rm{heat}}$}
\newcommand{\tcool}{$t_{\rm{cool}}$}
\newcommand{\rcool}{$r_{\rm{cool}}$}
\newcommand{\tcross}{$t_{\rm{cross}}$}
\newcommand{\tdyn}{$t_{\rm{dyn}}$}
\newcommand{\tkh}{$t_{\rm{KH}}$}
\newcommand{\tH}{$t_{\rm{H}}$}
\newcommand{\tvir}{\ifmmode{T_{\rm{vir}}}\else{$T_{\rm{vir}}$}\fi}
\newcommand{\mvir}{\ifmmode{M_{\rm{vir}}}\else{$M_{\rm{vir}}$}\fi}
\newcommand{\rvir}{\ifmmode{r_{\rm{vir}}}\else{$r_{\rm{vir}}$}\fi}
\newcommand{\rr}{$r_{200}$}
\newcommand{\lya}{Ly$\alpha$}
\newcommand{\jj}{\ifmmode{J_{21}}\else{$J_{21}$}\fi}
\newcommand{\flw}{\ifmmode{F_{LW}}\else{$F_{LW}$}\fi}
\newcommand{\kph}{\ifmmode{k_{\rm ph}}\else{$k_{\rm ph}$}\fi}
\newcommand{\tv}{$\langle T \rangle_{\rm v}$}
\newcommand{\tm}{$\langle T \rangle_{\rm m}$}
\newcommand{\msun}{{\rm\,M_\odot}} 
\newcommand{\lsun}{{\rm\,L_\odot}}
\newcommand{\zsun}{\ifmmode{\rm\,Z_\odot}\else{$\rm\,Z_\odot$}\fi}
\newcommand{\etal}{et al.\ }
\newcommand\tento[1]{$10^{#1}$}
\newcommand\step[1]{\textit{Step #1}.--}
\newcommand\halo[1]{\textit{Halo #1}.--}
\newcommand{\hi}{H {\sc i}}
\newcommand{\hii}{H {\sc ii}}
\newcommand{\hei}{He {\sc i}}
\newcommand{\heii}{He {\sc ii}}
\newcommand{\heiii}{He {\sc iii}}
\newcommand{\nhi}{\ifmmode{N_{\rm HI}}\else{$N_{\rm HI}$}\fi}
\newcommand\unit[1]{\; \textrm{#1}}
\newcommand{\pem}{\unit{s}^{-1} \unit{cMpc}^{-3}}
\newcommand{\music}{{\sc Music}}
\newcommand{\rockstar}{{\sc Rockstar}}

\newcommand{\refs}{(\textbf{Add references})}
\newcommand{\jhw}[1]{{\color{red} \bf (JHW: #1)}}


%%%%%%%%%%%%%%%%%%%%%%%%%%%%%%%%%%%%%%%%%%%%%%%%%%
%% Additions from AASTeX
%%%%%%%%%%%%%%%%%%%%%%%%%%%%%%%%%%%%%%%%%%%%%%%%%%

%\newcommand\ion[2]{#1$\;${\small\rmfamily\@Roman{#2}}\relax}
\def\eps@scaling{1.0}% 
\newcommand\epsscale[1]{\gdef\eps@scaling{#1}}% 
\newcommand\plotone[1]{% 
 \centering 
 \leavevmode 
 \includegraphics[width={\eps@scaling\columnwidth}]{#1}% 
}% 
\newcommand\plottwo[2]{% 
 \centering 
 %\leavevmode 
 %\columnwidth=.5\columnwidth 
 \includegraphics[width={\eps@scaling\columnwidth}]{#1}% 
 \hfil 
 \includegraphics[width={\eps@scaling\columnwidth}]{#2}% 
}% 

% \newcommand{\nat}{Nature}
% \newcommand{\aj}{AJ}
% \newcommand{\apj}{ApJ}
% \newcommand{\apjl}{ApJL}
% \newcommand{\apjs}{ApJS}
% \newcommand{\mnras}{MNRAS}
% \newcommand{\aap}{A\&A}
% \newcommand{\pasj}{PASJ}
% \newcommand{\pasp}{PASP}
% \newcommand{\araa}{ARA\&A}
% \newcommand{\bain}{Bull.~Astron.~Inst.~Netherlands}
% \newcommand{\memsai}{Mem.~Soc.~Astron.~Italiana}
% \newcommand{\physrep}{Physics Reports}
% \newcommand{\prd}{Phys.~Rev.~D}
% \newcommand{\jcap}{JCAP}

\hyphenation{sSFR sSFRs}

\makeatletter
\newcommand*{\rom}[1]{\expandafter\@slowromancap\romannumeral #1@}
\makeatother

% Please keep new commands to a minimum, and use \newcommand not \def to avoid
% overwriting existing commands. Example:
%\newcommand{\pcm}{\,cm$^{-2}$}	% per cm-squared

%%%%%%%%%%%%%%%%%%%%%%%%%%%%%%%%%%%%%%%%%%%%%%%%%%

%%%%%%%%%%%%%%%%%%% TITLE PAGE %%%%%%%%%%%%%%%%%%%

% Title of the paper, and the short title which is used in the headers.
% Keep the title short and informative.
\title[Where Do Pop III Stars Form?]{Where Do Population III Stars Form?}

% The list of authors, and the short list which is used in the headers.
% If you need two or more lines of authors, add an extra line using \newauthor
\author[Danielle Skinner et al.]{
Danielle Skinner,$^{1}$\thanks{E-mail: drenniks@gatech.edu}
John H. Wise,$^{1}$
\\
% List of institutions
$^{1}$Center for Relativistic Astrophysics, Georgia Institute of Technology, 
Atlanta, GA 30332, USA\\
}

% These dates will be filled out by the publisher
%\date{Accepted XXX. Received YYY; in original form ZZZ}

% Enter the current year, for the copyright statements etc.
\pubyear{2018}

% Don't change these lines
\begin{document}
\label{firstpage}
\pagerange{\pageref{firstpage}--\pageref{lastpage}}
\maketitle

% Abstract of the paper
\begin{abstract}
This is a simple template for authors to write new MNRAS papers.
The abstract should briefly describe the aims, methods, and main results of the 
paper.
It should be a single paragraph not more than 250 words (200 words for Letters).
No references should appear in the abstract.
\end{abstract}

% Select between one and six entries from the list of approved keywords.
% Don't make up new ones.
\begin{keywords}
keyword1 -- keyword2 -- keyword3
\end{keywords}

%%%%%%%%%%%%%%%%%%%%%%%%%%%%%%%%%%%%%%%%%%%%%%%%%%

%%%%%%%%%%%%%%%%% BODY OF PAPER %%%%%%%%%%%%%%%%%%

\section{Introduction}

\noindent Here are some relevant papers that studied the minimum halo
mass with simulations or ones that developed an analytical model of
the halo mass.  We will probably cite most of these in the
introduction.

\lli Semi-analytic papers: \citep{Tegmark97, Trenti09, Visbal18,
  Mebane18, Griffen18}

\lli Simulation papers: \citep{Machacek01, Yoshida03, Wise07_UVB,
  OShea08, Muratov13}

\lli {\it Note:} I looked at which papers cited the \citet{Machacek01}
paper to look for recent papers.  I didn't find any in the past five
years that used simulations to study the minimum halo mass.

\medskip

The first generation of stars transformed a cold and dark universe
by illuminating and enriching up their neighborhoods with radiation
and heavy elements. The formation of these first stars is a crucial step in the cosmological evolution of the universe because the metals and feedback they deliver to their local environments is necessary for further star production and chemical enrichment. Without these initial stars, heavier metals would not have been produced, and a different universe would be observed than what is observed today. These stars are, by definition, metal-free (Population III; Pop III) and are thought to be generally massive. A substantial fraction of these stars will generate prodigious amounts of ionizing photons and will end their lives in some form of a supernova \citep[e.g.][]{Schaerer02, Heger02}. The supernova will spread the enriched guts of the Pop III star out across it's local environment, providing the area with chemicals the universe has not yet seen. 

Once a halo becomes chemically enriched by the death of it's Pop III stars, then by definition, it can no longer form more Pop III stars. This marks the end of metal-free star formation in that pregalactic object. Understanding the mixing of metals into the environment of these halos is necessary to constrain the reach of these metals and the effects they have for future star formation. Numerical studies have shown that turbulence within halos can mix metals well down to their resolution limit \citep[][and more]{Wise08_Gal, Greif10}. Stars will continue to form in halos, but with an increase metal abundance, these stars are now labelled as Population II, or Pop II, stars. Pop II stars are considered metal-poor, relative to their metal-free Pop III progenitors, and are observable in the universe. Understanding the formation and chemical abundance of Pop II stars can provide more evidence and insight for the earlier generation of Pop III stars.

Given the paradigm of hierarchical structure formation, halos grow through smooth accretion and successive mergers of smaller halos. But in which halos do these first generations of stars form? Their formation rates and locations are important to constrain because they influence the very beginnings of galaxy formation and cosmic reionization. Learning about their host halos can also guide future telescopic surveys to observe Pop III galaxies which may lead to direct observational evidence for Pop III stars.
Without metals and dust to facilitate efficient radiative cooling, Pop III stars rely on \hh{} formation to cool the gas. These molecules are however fragile to dissociation from Lyman-Werner (LW) radiation in the energy range 11.2--13.6~eV, through the Soloman process :
\begin{equation}
	H_{2} + \gamma \rightarrow  H + H
\end{equation}
Diffuse gas is optically thin to LW radiation, thus a background builds over time and can suppress \hh{} formation, prohibiting Pop III star formation. Furthermore, nearby sources of LW radiation can dissociate \hh{} in minihalos. This process can be counteracted with a sufficient amount of \hh{} already present within a halo, via \hh{} self-shielding. Halos with an \hh{} column density of N$_{H_{2}}$ $\geq$ 10$^{14}$ cm$^{-2}$ can suppress the photodissociation of \hh{} by LW photons, giving the halo a chance to form Pop III stars. Other sources of the suppression of Pop III star formation are streaming baryonic velocities, arising from recombination, and dynamical heating, occuring during the virialization of halos. 

Previous work have mainly investigated and established the lower limit for a halo to host Pop III stars, or for a halo to collapse. However, subsequent simulations have shown that they don't necessarily form at this minimum due to other physical processes (mentioned above) playing a role.

In this paper, we focus on the distribution of host halo masses, not just the minimum, and its dependence on redshift and the LW background, augmenting results from prior work. We aim to provide further insight into the host halos of Pop III stars. In \S 2 the methods of the simulation, implementation of star formation, feedback, and \hh{} self-shielding is described. In \S 3 we present the results of our analysis. In \S 4 we discuss the results in more detail and compare with previous work. In \S 5 we conclude our discussion with a summary of our results and the implications therein.

%====================================================================
\section{Methods}
%====================================================================
\subsection{Simulation setup}
%====================================================================
This simulation is an expansion of the analysis done on the 
simulations from Wise et al. (2014, hereafter W14) and Wise et al. 
(2012b, hereafter W12). W12 focused on the role of radiation pressure in simulated halos and W14 was an extension of the W12 paper, specifically looking at the simulation including radiation pressure. While the cosmological parameters are updated to the 2013 Planck release \citet{Planck13_Cosmo} for this simulation, much of the physics is the same as W12 and W14. For more detailed descriptions, refer to W12 and W14.

This simulation was run with an adaptive mesh refinement (AMR) code 
\enzo{} \citep{Enzo} and \yt{} \citep{yt_full_paper} for visualization. \enzo{} uses an N-body adaptive particle-mesh solver \citep{Efstathiou85, Couchman91, BryanNorman1997} to follow the dark matter (DM) dynamics. We use the nine-species (H \rom{1}, H \rom{2}, He \rom{1}, He \rom{2}, He \rom{3}, e$^{-}$, H$_{2}$, H$_{2}^{+}$, H$^{-}$) non-equilibrium chemistry model \citep{Abel97, Anninos97}. 

We simulate a 1 Mpc$^{3}$ comoving box with a 256$^{3}$ base grid 
resolution. We have a maximum refinement level of 12 which provides a maximal comoving resolution of $\sim$1 pc. The simulation is 
initialized with \music{} \citep{Hahn11_MUSIC} at z = 130 and use the cosmological parameters from the Planck collaboration best fit 
\citet{Planck13_Cosmo}: $\Omega_{M}$ = 0.3175, $\Omega_{\Lambda}$ = 
0.6825, $\Omega_{DM}$ = 0.2685, and h = 0.6711, with the variables 
having their usual definitions. The simulation is run until z = 9.32. In this paper, we focus only on outputs where new Pop III stars are born, from z = 27.23 to 9.39. 

A time-dependent Lyman-Werner (LW) optically thin radiation background is applied in the simulation. This was fit in W12 (see Eqn. 16) and is consistent with the values in \citet{Trenti09_SFR}. The background evolution of the specific intensity takes the following form:
\begin{equation}
	log_{10}(J_{21}) = A + Bz - Cz^{2} + Dz^{3} - Ez^{4}
\end{equation}
where (A, B, C, D, E) = (-2.567, 0.4562, - 0.02680, 5.882 $\times$ 10$^{-4}$, - 5.056 $\times$ 10$^{-6}$) and J$_{21}$ is the specific intensity in units of 10$^{-21}$ erg s$^{-1}$ cm$^{-2}$ Hz$^{-1}$ sr$^{-1}$. 

Adaptive ray tracing \citep{Abel02_RT, Wise11_Moray} is used to evolve the radiation field. Centered on each Pop II and Pop III star, we model the H$_{2}$ dissociating radiation field with an optically thin, inverse square profile. These LW sources are added on top of the background intensity described above. (When to describe how we caculate the total Lyman-Werner flux of the halos?)
%====================================================================
\subsection{Star formation}
%====================================================================
In this section, we will discuss how we implement Pop III and Pop II 
star formation. No other types of stars are considered. For further 
details, refer to W14 and W12. 

%====================================================================
\subsubsection{Pop III Star Formation and Feedback }
%====================================================================
The implemented Pop III star formation model is the same as in Wise \& Abel (2008). Each star particle represents a single star, and is formed in a cell when the following criteria are met: 
\begin{enumerate}
	\item Z $\leq$ 6.85 $\times$ 10$^{-8}$

	\item particle overdensity of 1 $\times$ 10$^{6}$ 
		particles/cm$^{3}$

	\item converging gas flow, $\nabla \cdot \mathbf{v_{gas}}$ < 0 

	\item molecular hydrogen fraction, f$_{H2}$ > 0.001
\end{enumerate}

If within 1 pc, multiple cells meet this criteria, then a single Pop III star forms at the center of mass of these cells. The mass of the Pop III star is randomly sampled from an initial mass function (IMF) of the form:
\begin{equation}
	f(log_{M})dM = M^{-1.3}exp \Big[-\left( \frac{M_{char}}{M}\right)^{1.6} \Big]dM
\end{equation}
where M$_{char}$ = 20 M$_{\odot}$. This characteristic mass is different from W12 and W14 (would like to find a citation for this number, having a hard time tracking down some John had suggested). This results in an exponential cutoff before M$_{char}$ and a Salpeter IMF after. Once the cell meets the above criteria and the mass of the star is randomly determined, an equal amount of gas is removed from the grid in a sphere which contains twice the stellar mass that is centered on the new star particle. The new Pop III star then gains the mass-weighted velocity of the gas contained within that sphere.  

We use the mass-dependent hydrogen ionising Lyman-Werner (LW) luminosities and lifetimes from Schaerer (2002) and a mass-independent photon energy of E$_{ph}$ = 29.6 ev, which is appropriate for the approximately constant surface temperature of Pop III stars, at 10$^{5}$ K. A Type II supernova results if the Pop III star dies with a mass between $\leq$ M$_{\ast}$/M$_{\odot}$ $\leq$ 40 M$_{\odot}$. A pair instability supernova (PISNe) results if the star dies with a mass between 140 $\leq$ M$_{\ast}$/M$_{\odot}$ $\leq$ 260 M$_{\odot}$. Outside of these mass ranges, the Pop III star dies as a black hole. In this simulation, no black hole physics are implemented, so the Pop III star simply turns into a dark matter particle. 
From 11 $\leq$ M$_{\ast}$/M$_{\odot}$ < 20 M$_{\odot}$, the star dies as a normal Type II supernova with an energy of 10$^{51}$ erg. From 20 $\leq$ M$_{\ast}$/M$_{\odot}$ $\leq$ 40 M$_{\odot}$, the star dies as a hypernova with energies taken from \citet{Nomoto06}.
The PISNe have explosion energies from \citet{2002ApJ...567..532H}, of which the analytic function fit to their models can be seen in Eq. 12 of W12. 

%====================================================================
\subsubsection{Pop II  Star Formation and Feedback}
%====================================================================
The Pop II star formation model is the same as Wise \& Cen (2009), which is equivalent to the above criteria for Pop III star formation but without the molecular hydrogen fraction requirement, and with a metallicity greater than the above value. Star formation is only allowed for gas with temperatures T < 1000 K, to ensure that the volume is cooling. Contrary to Pop III stars representing single star particles, Pop II star particles are modelled as clusters of Pop II stars. The mass of these clusters is determined from a Salpeter IMF with a minimum mass of M$_{min}$ = 1000 M$_{\odot}$. 

The lifetime of these particles is 20 Myr, during which they radiate 6000 hydrogen ionizing photons per stellar baryon. The photons exhibit a monochromatic spectrum with an energy of 21.6 eV. After living for 4 Myr, the Pop II stars generate supernova energies of 6.8 $\times$ 10$^{48}$ erg s$^{-1}$ M$_{\odot}^{-1}$.

%====================================================================
\subsection{H2 Shielding}
%====================================================================
We use the Sobolev-like approximation from \citet{Wolcott11} to model H2 self-shielding. To determine the H2 shielding factor, the column density of H2 needs to be calculated:
\begin{equation}
	N_{H_{2}} = n_{H_{2}}L_{char}
\end{equation}
Where n$_{H_{2}}$ is the number density of H2 and L$_{char}$ is a characteristic length over which n$_{H_{2}}$ is assumed to be constant. The method employed in this simulation is to define the characteristic length as:
\begin{equation}
	L_{char} = \frac{\rho}{|\nabla \rho|}
\end{equation}
This Sobolev-like method determines the length over which gas with density $\rho$ is diminished. The most accurate, non-ray tracing, method is to use a single Sobolev-like length determined from the mean Sobolev-like length over all directions. 

Upon numerically calculating the shielding factor of H2 for simulated protogalaxies, Wolcott-Green et al. determine that a slight adjustment to the shielding factor from \citet{Draine96} is sufficient to account for inaccuracies at higher temperatures. The implemented shielding factor is then (see Eqn. 10 in \citet{Wolcott11}):
\begin{equation}
	\begin{multlined}
	f_{sh}(N_{H2}, T) = \frac{0.965}{(1+x/b_{5})^{1.1}} + \frac{0.035}{(1+x)^{0.5}}  \times \\ \exp [-8.5 \times 10^{-4} (1+x)^{0.5}]
	\end{multlined}
\end{equation}
Here, x $\equiv$ N$_{H_{2}}$/5 $\times$ 10$^{14}$ cm$^{-2}$, b$_{5}$ $\equiv$ b/10$^{5}$ cm s$^{-1}$, and b is the Doppler broadening parameter. N$_{H_{2}}$ is then calculated as described above.


%====================================================================
\section{Results}
%====================================================================
Temporary list of figures:
\begin{enumerate}
\item hmf.png
\item Mean Halo Mass vs. Redshift - plotted with Machacek relation - mean\_mass\_errorb\_fix.png
\item Jlw/J21 total vs. Halo Mass colored by redshift - plotted with Machacek relation - jlw\_mass\_machacek\_total.png
\item Percentage of Halos Below Machacek - percent\_below\_dz1.png 
\item Jlw/J21 vs. Halo Mass colored by redshift - plotted with Machacek relation ??  - jlw\_mass\_machacek.png
\item Halo mass vs. redshift colored by total Jlw - plotted with background Machacek relation ?? - mass\_redshift\_machacek\_clocal.png
\item SFRD of PopIII Stars - pop3\_SFR\_bar.png
\item Total \# of PopIII Stars vs. Halo mass 2d histogram - tot\#p3\_halomass\_sidehist.png
\item Total mass of PopIII stars vs. halo mass 2d histogram - plotted with SFE - totp3mass\_halomass\_sidehist.png

\end{enumerate}

%====================================================================
\subsection{Population III host halo masses}
%====================================================================

%====================================================================
\subsection{Multiple stellar systems}
%====================================================================

%====================================================================
\subsection{Population III stars in atomic cooling halos}
%====================================================================

%====================================================================
\section{Discussion}
%====================================================================

%====================================================================
\subsection{Variations in the halo masses at collapse}
%====================================================================

\li Halos can form metal-free stars again if their progenitors had
only formed stars that did not go supernova

\li Dynamical heating \citep{Yoshida03}

\li Temporal fluctuations in the local Lyman-Werner radiation

%====================================================================
\subsection{Comparison to previous work}
%====================================================================

\li Semi-analytic papers: \citep{Tegmark97, Trenti09, Visbal18,
  Mebane18, Griffen18}

\li Simulation papers: \citep{Machacek01, Yoshida03, Wise07_UVB,
  OShea08, Muratov13}

\li We don't have to compare with all of the semi-analytic papers, but
it will be good to compare with all of the simulation papers.

%====================================================================
\subsection{Caveats}
%====================================================================

\li Optically-thin Lyman-Werner radiation field from point sources \citep{Schauer17}

\li No streaming velocities that suppresses star formation at very
high ($z \ga 20$) redshifts \citep{Tselia11, Greif11_Delay, Naoz12, OLeary12}

\li Uncertainties from the assumed Pop III initial mass function 

%====================================================================
\section{Conclusions}
%====================================================================

%====================================================================
\section*{Acknowledgements}
%====================================================================

JHW is supported by National Science Foundation grants AST-1614333 and
OAC-1835213, NASA grant NNX17AG23G, and Hubble theory grant
HST-AR-14326.  We thank the support staff at Georgia Tech's PACE,
where we ran this simulation.  The freely available plotting library
{\sc matplotlib} \citep{matplotlib} was used to construct numerous
plots within this paper. Computations and analysis described in this
work were performed using the publicly-available \enzo{} and \yt{}
codes, which is the product of a collaborative effort of many
independent scientists from numerous institutions around the world.

%%%%%%%%%%%%%%%%%%%%%%%%%%%%%%%%%%%%%%%%%%%%%%%%%%

%%%%%%%%%%%%%%%%%%%% REFERENCES %%%%%%%%%%%%%%%%%%

% The best way to enter references is to use BibTeX:

\bibliographystyle{mnras}
\bibliography{jwise} % if your bibtex file is called example.bib


% Alternatively you could enter them by hand, like this:
% This method is tedious and prone to error if you have lots of references
%\begin{thebibliography}{99}
%\end{thebibliography}

%%%%%%%%%%%%%%%%%%%%%%%%%%%%%%%%%%%%%%%%%%%%%%%%%%

%%%%%%%%%%%%%%%%% APPENDICES %%%%%%%%%%%%%%%%%%%%%

\appendix

%%%%%%%%%%%%%%%%%%%%%%%%%%%%%%%%%%%%%%%%%%%%%%%%%%


% Don't change these lines
\bsp	% typesetting comment
\label{lastpage}
\end{document}

% End of mnras_template.tex